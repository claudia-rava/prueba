% Actualizado abril 2017 latex 
\documentclass{article}
%\documentclass[11pt,a4paper]{article}

\usepackage[a4paper, total={6.5in, 9in}]{geometry}
\usepackage{comment}
%\textwidth15cm
%\hoffset -1 cm
\begin{document}
%\hsize=6.5truein
%\hsize=6.5truein
%\vsize=9.0truein
\def\sm{\smallskip}
\def\ms{\medskip}
\def\bs{\bigskip}
\def\noi{\noindent}
\bigskip
\bigskip
\noi {\bf CURRICULUM VITAE NORMALIZADO}

\noi{ \vrule width 14cm height 0.8pt}
\bigskip
\bigskip

\noindent {\bf 01 - Antecedentes Personales:\/}

\bigskip

\noindent Apellido: RAVAZZOLI

\noindent Nombres: Claudia Leonor

\noindent Lugar de Nacimiento: La Plata, Provincia de Buenos Aires.

\noindent Nacionalidad: Argentina

\noindent Fecha de Nacimiento : 30 de abril de 1965 \qquad\qquad\quad \ \
Estado Civil : casada, 1 hijo

\noindent Documento de Identidad: DNI \qquad\quad Nro.: 17.569.146

\noindent C\'edula de Identidad: Nro. 10.737.411 \qquad\qquad \qquad\quad \ \
Polic\'\i a:  Federal

\noindent Domicilio Real: Calle: Diag. 75 \quad Nro: 928 \quad\qquad\qquad
Localidad: La
Plata

\noindent C.P.: 1900 Provincia: Buenos Aires

\noindent Tel\'efono: (0221)4531540  \quad Fax:(0221) 4236591

\noi e-mail: claudia@fcaglp.fcaglp.unlp.edu.ar

\noindent Domicilio de Notificaciones dentro del Radio
Urbano de La Plata (Art. 20 Ord. 101)

\noindent Calle: Diag. 75 \quad Nro: 928 

\noindent Tel\'efono: (0221)4531540      \quad Fax:(0221)4236591
\bigskip

\noindent {\bf 02 - Estudios Realizados y T\'\i tulos Obtenidos:\/}
\bigskip

\noindent Universitarios:

\noindent De grado: Geof\'\i sica, obtenido en la Universidad Nacional de
La Plata el 09/06/88.
\sm

\noindent De Post-Grado: Doctora en Geof\'\i sica, obtenido en la misma
Universidad el 30/11/95.

Doctorado categor\'\i a B, acreditado por CONEAU por Resoluci\'on 211/99.

Doctorado categor\'\i a A, acreditado por CONEAU por Resoluci\'on 164/12.
\sm

%\noindent Otros Estudios Superiores:

\bigskip
\bigskip
\noindent {\bf 03 - Tesis de Doctorado o Maestr\'\i a:\/}

\bigskip

\noindent T\'\i tulo: ``Modelado de fen\'omenos de propagaci\'on de ondas
en medios disipativos''
\sm
\noindent Realizada en: Departamento de Geof\'\i sica Aplicada, Facultad
de Ciencias Astron\'omicas y Geof\'\i sicas (F.C.A.G.),
Universidad Nacional de La Plata (U.N.L.P).
\sm

\noindent Director de Tesis: Dr. Juan Enrique Santos

\noindent Calificaci\'on: Sobresaliente (10).

%\newpage
\bs

\noindent {\bf 04 - Becas:\/}
\bigskip

\noindent Tipo: Beca de Estudio (interna).

\noindent Fecha Inicio: 25/05/89 \qquad\qquad Fecha terminaci\'on :
31/03/91.

\noindent Lugar: Departamento de Geof\'\i sica Aplicada,
Facultad de
Ciencias Astron\'omicas y Geof\'\i sicas, Universidad Nacional de La
Plata.

\noindent Instituci\'on Otorgante: Comisi\'on de Investigaciones
Cient\'\i ficas de la Provincia de Buenos Aires.

\noindent Por Concurso:   \quad Si

\medskip
\noindent Tipo: Beca de Perfeccionamiento (interna).

\noindent Fecha Inicio: 01/04/91  \qquad\qquad
Fecha terminaci\'on: 31/03/93.

\noindent Lugar: Idem anterior.

\noindent Instituci\'on Otorgante: Idem anterior.

\noindent Por Concurso:   \quad Si


\medskip
\noindent Tipo: Pr\'orroga Especial de Beca de Perfeccionamiento.

\noindent Fecha Inicio: 01/04/93  \qquad\qquad
Fecha terminaci\'on: 01/10/93.

\noindent Lugar: Idem anterior.

\noindent Instituci\'on Otorgante: Idem anterior.

\noindent Por Concurso:   \quad Si

\bigskip
\noindent {\bf 05 - Cursos de Perfeccionamiento Seguidos:\/}

\bigskip
\noindent Nombre : ``Algebra Lineal Computacional''

\noindent Duraci\'on: marzo de 1989

\noindent Asisitido o Aprobado: Aprobado

\noindent Instituci\'on: F.C.A.G.

\noindent Carga Horaria: 18 hs.
\medskip

\noindent Nombre: ``Geolog\'\i a del Subsuelo Aplicada a la Industria
Petrolera''

\noindent Duraci\'on: del 2 al 11 de Octubre de 1990

\noindent Asisitido o Aprobado: Aprobado

\noindent Instituci\'on: Facultad de Ciencias Naturales y Museo y
C\'\i a. Schlumberger

\noindent Carga Horaria: 40 hs.

\medskip
\noindent Nombre: ``Redes Acad\'emicas''

\noindent Duraci\'on: Noviembre de 1990

\noindent Asistido o aprobado: asistido

\noindent Instituci\'on: F.C.A.G

\noindent Carga Horaria: 18 hs.

\medskip

\noindent Nombre: ``An\'alisis Matem\'atico IV''

\noindent Duraci\'on: materia anual ciclo lectivo 1991

\noindent Asistido o aprobado: aprobado (como Seminario para el Doctorado)

\noindent Instituci\'on: F.C.A.G.

\noindent Carga Horaria: 5 hs. por semana.

\medskip

\noindent Nombre: ``Teor\'\i a de Modos Normales'' (seminario)

\noindent Duraci\'on: Octubre - Noviembre de 1991

\noindent Asistido o aprobado: asistido

\noindent Instituci\'on: F.C.A.G

\noindent Carga Horaria: 30 hs.

\medskip
\noindent Nombre: ``Principles of Waterflooding, (short seminar)''

\noindent Duraci\'on: medio d\'\i a (31/08/93)

\noindent Asistido o aprobado: asistido

\noindent Instituci\'on: Hycal Sismos Technologies e Instituto Argentino
del Petr\'oleo

\noindent Carga Horaria: 4 hs.
\medskip

\noindent Nombre: ``A presentation on Formation Damage,  (short  seminar)''

\noindent Duraci\'on: medio d\'\i a (31/08/93)

\noindent Asistido o aprobado: asistido

\noindent Instituci\'on: Hycal Sismos Technologies e Instituto Argentino
del Petr\'oleo

\noindent Carga Horaria: 4 hs.
\medskip

\noindent Nombre: ``Curso de Unix''

\noindent Duraci\'on: Octubre de 1993

\noindent Asistido o aprobado: asistido

\noindent Instituci\'on: F.C.A.G

\noindent Carga Horaria:

\medskip
\noindent Nombre: ``T\'ecnicas en pozo entubado: Perfilaje y Ensayos''

\noindent Duraci\'on: del 21 al 23 de Marzo de 1994.

\noindent Asistido o aprobado: asistido

\noindent Instituci\'on: C\'\i a. Schlumberger

\noindent Carga Horaria: aprox. 15 hs.

\medskip

\noindent Nombre: ``Nuevas t\'ecnicas de s\'\i smica de pozo''

\noindent Duraci\'on: del 22 al 24 de Agosto de 1994

\noindent Asistido o aprobado: asistido

\noindent Instituci\'on: C\'\i a. Schlumberger

\noindent Carga Horaria: aprox. 15 hs.

\medskip
\noindent Nombre: ``El Mapa Log\'\i stico: una introducci\'on al Caos''

\noindent Duraci\'on: Agosto - Septiembre de 1994

\noindent Asistido o aprobado: asistido

\noindent Instituci\'on: F.C.A.G.

\noindent Carga Horaria: 9 hs.

\medskip
\noindent Nombre: ``Resoluci\'on Num\'erica de Ecuaciones Diferenciales''

\noindent Duraci\'on: segundo cuatrimestre de 1994

\noindent Asistido o aprobado: asistido

\noindent Instituci\'on: F.C.A.G.

\noindent Carga Horaria: 4 hs. semanales
\medskip

\noindent Nombre: ``Los m\'etodos potenciales en la interpretaci\'on
geol\'ogica - geof\'\i sica integrada''

\noindent Duraci\'on: del 7 al 12 de Octubre de 1996.

\noindent Asistido o aprobado: aprobado.

\noindent Instituci\'on: F.C.A.G.

\noindent Carga Horaria:  35 hs.


\medskip
\noindent Nombre: ``El Geoide desde el punto de vista gravim\'etrico''

\noindent Duraci\'on: segundo cuatrimestre de 1996

\noindent Asistido o aprobado: asistido.

\noindent Instituci\'on: F.C.A.G.

\noindent Carga Horaria:  4 hs. semanales.

\medskip
\noindent Nombre: ``Atenuaci\'on S\'\i smica''

\noindent Duraci\'on: del 13/12/99 al 17/12/99 

\noindent Asistido o aprobado: asistido.

\noindent Instituci\'on: F.C.A.G.

\noindent Carga Horaria:  12 hs.

\bigskip
\noindent {\bf 06 - Distinciones - Premios:\/}
\medskip

Premio a la Labor Cient\'\i fica, Tecnol\'ogica y Art\'\i stica de la Universidad 
Nacional de La Plata 2017, en la categor\'\i a Investigador Formado, Resol. UNLP 1297/17.


\bigskip
\noindent {\bf 07 - Antecedentes Docentes y de Investigaci\'on:\/}
\medskip

\noindent {\bf 07.1 - En Grado:\/}
\medskip

\noindent Cargo: Ayudante de 2da. Ordinario ad-honorem

\noindent Dedicaci\'on: Simple

\noindent C\'atedra: Gravimetr\'\i a

\noindent Periodicidad: anual

\noindent Per\'\i odo: desde el 02/03/88 hasta el 31/01/89
\medskip

\noindent Cargo: Ayudante Diplomado Interino

\noindent Dedicaci\'on: Simple

\noindent C\'atedra: Gravimetr\'\i a

\noindent Periodicidad: anual

\noindent Per\'\i odo: desde el 01/02/89  hasta  el 31/08/90
\medskip

\noindent Cargo: Ayudante Diplomado Interino

\noindent Dedicaci\'on: Semi-exclusiva

\noindent C\'atedra: Gravimetr\'\i a

\noindent Periodicidad: anual

\noindent Per\'\i odo: desde el 01/09/90 hasta  el 31/03/92
\medskip

\noindent Cargo: Ayudante Diplomado Interino

\noindent Dedicaci\'on: Exclusiva

\noindent C\'atedra: Gravimetr\'\i a

\noindent Periodicidad: anual

\noindent Per\'\i odo: desde el 01/04/92 hasta  el 10/08/92
\medskip

\noindent Cargo: Jefe de Trabajos Pr\'acticos Interino

\noindent Dedicaci\'on: Exclusiva

\noindent C\'atedra: Gravimetr\'\i a

\noindent Periodicidad: anual

\noindent Per\'\i odo: desde el 11/08/92 hasta el 15/12/93
\medskip

\noindent Cargo: Jefe de Trabajos Pr\'acticos Ordinario

\noindent Dedicaci\'on: Exclusiva

\noindent C\'atedra: Gravimetr\'\i a

\noindent Periodicidad: semestral (por modificaci\'on de plan de
estudio)

\noindent Per\'\i odo: desde el 16/12/93 hasta el 30/04/97.

\medskip

\noindent Cargo: Profesor Adjunto Interino

\noindent Dedicaci\'on: Exclusiva

\noindent C\'atedra: M\'etodos Potenciales de Prospecci\'on

\noindent Periodicidad: semestral

\noindent Per\'\i odo: desde el 01/05/97 al 02/10/2000.

\medskip

\noindent Cargo: Profesor Adjunto Ordinario

\noindent Dedicaci\'on: Exclusiva

\noindent C\'atedra: M\'etodos Potenciales de Prospecci\'on

\noindent Periodicidad: semestral

%\noindent Per\'\i odo: desde el 28/08/2000 al presente.
\noindent Per\'\i odo: desde el 03/10/2000 al 02/11/2010.
\medskip

\noindent {\bf Cargo Actual}: Profesor Asociado Ordinario

\noindent Dedicaci\'on: Exclusiva

\noindent C\'atedra: M\'etodos Potenciales de Prospecci\'on

\noindent Periodicidad: semestral

\noindent Per\'\i odo: 02/11/2010 al 02/11/2018. Expte. 1100-611/10,
Resol. 316 17/11/2010.

%designada el 06/08/2010 Prof. Asoc. Interino (a confirmar x Cons. Sup.) 
% 02/11/2010 desiganda en el cargo ordinario
% 17/09/10 envie decl jurada al CONICET declarando el cargo interino a partir del
%01-08-10. En info mayo 2011 mando la declaracion del cargo ordinario

\bigskip
\medskip

\noindent {\bf 07.2 - Post--Grado:\/}
\bigskip

Docente responsable del Seminario de Postgrado cuatrimestral ``Fundamentos y Aplicaciones 
de la F\'\i sica de Rocas'', acreditado en la FCAG (Expte. 1100-867/10) (4 cr\'editos), 
dictado a partir de septiembre de 2010 (hasta 2015).
\medskip

Integrante del cuerpo docente de la {\it Especializaci\'on en Geociencias
de Exploraci\'on y Desarrollo de Hidrocarburos}, Facultad de Cs. Naturales
y Museo, desde 2015. Dictado del curso sobre {\it M\'etodos Potenciales y Electromagn\'eticos
de Prospecci\'on} (20 hs c\'atedra, con examen final).
\medskip

Docente responsable de  \emph{Teor\'\i a y Modelos de F\'\i sica de Rocas}, materia de posgrado acreditado en la FCAG (Expte. 1100-2839/17) (5 cr\'editos), 
dictado a partir de 2017.
\bigskip

\noindent {\bf 07.3 - Categor\'\i a de docente - investigador Programa de Incentivos}

\bigskip
\noindent Fecha y categor\'\i a de ingreso: 01/05/94. Categor\'\i a D.

\noindent Promoci\'on a categor\'\i a  III (Comisi\'on F\'\i sica,
Astronom\'\i a y Geof\'\i sica), el 11/10/1999. Recategorizada con el mismo nivel el 09/02/2005.

\noi {\bf Situaci\'on actual}: categor\'\i a II (Comisi\'on F\'\i sica,
Astronom\'\i a y Geof\'\i sica), obtenida el 09/12/2010.
% segun el reglamento hay que tener 
% 4 a�os en la categor�a anterior antes de recategorizar.  
%\noindent Lugar de trabajo: Departamento de Geof\'\i sica Aplicada (F.C.A.G.,
%U.N.L.P.)
% No se si me presento al llamado a categorizacion de marzo 2015

\bigskip
\noindent {\bf 08 - Cargos y Funciones Desempe\~nados:\/}
\bigskip

\noindent {\bf 08.1 - Universitarios}
\medskip
\begin{itemize}
\item Consejera Acad\'emica Suplente y Titular por el Claustro de
Graduados de la Facultad de Cs. Astron\'omicas y Geof\'\i sicas  desde
1992  hasta  abril  de  1997.  Miembro  de  las  Comisiones  de  Grado
Acad\'emico y de Ense\~nanza de  la misma Facultad desde 1992  y desde
1996, respectivamente hasta la misma fecha.

\item Secretaria de Postgrado F.C.A.G. desde el 01/09/98 al 31/03/99. 
Resol.81, Expte. 1100-3864/98.

\item Coordinadora representante de la F.C.A.G. para la Comisi\'on Central de
Evaluaci\'on Institucional de  la U.N.L.P. desde el 29/09/99 hasta el
01/05/2000.  

\item Consejera Superior Suplente y Titular por el Claustro de
Profesores de la Facultad de Cs. Astron\'omicas y Geof\'\i sicas  desde
2001 hasta 2004. Miembro  Titular  por el claustro de profesores de la F.C.A.G.
 en la Comisi\'on  de Ciencia y T\'ecnica de la Universidad Nacional de La Plata 
durante el per\'\i odo 2003-2004.


\item Consejera Acad\'emica Suplente por el Claustro de
Profesores de la Facultad de Cs. Astron\'omicas y Geof\'\i sicas per\'\i odo 
abril de 2004 - abril de 2007. Miembro  Titular por el claustro de profesores 
de la Comisi\'on  de Ense\~nanza y Miembro  Suplente de la Comisi\'on de
 Investigaciones Cient\'\i ficas y de Grado Acad\'emico de la F.C.A.G. 
durante el per\'\i odo abril de 2004 - abril de 2007.

\item Integrante del Comit\'e Acad\'emico para la carrera de Doctorado en 
Geof\'\i sica
de la F.C.A.G. a partir del 18/08/05 (Expte. 1100-2663,cde. 5-05). 2. 
Designaci�n ratificada a partir de agosto 2011, (Expte. 1100-1412, resoluci�n 72/11). 

\item Miembro Titular de la Comisi\'on Instructora para Juicios Acad\'emicos de la F.C.A.G. a partir del 28/09/08 (Resol. 243, Expte. 1100-2294/08), hasta junio de 2010.

\item Integrante por el claustro de profesores como miembro externo en la Comisi\'on de
 Investigaciones Cient\'\i ficas de la F.C.A.G. para la evaluaci\'on de Informes de Mayor Dedicaci\'on desde agosto de 2010 hasta marzo de 2011.

\item Consejera Acad\'emica Titular por el Claustro de
Profesores de la Facultad de Cs. Astron\'omicas y Geof\'\i sicas per\'\i odo 
abril de 2014 - abril de 2018. Miembro  Titular por el claustro de profesores 
de la Comisi\'on  de Interpretaci\'on, Reglamento y Finanzas de la F.C.A.G. 
durante el mismo per\'\i odo.

\item Miembro de la Comisi\'on Asesora T\'ecnica correspondiente a Subsidios para 
Viajes y/o Estad\'\i as 2016-2017  de la UNLP. Abril de 2016.

\item
Miembro de la Comisi\'on Asesora T\'ecnica correspondiente a Subsidios para J\'ovenes Investigadores de la UNLP. Septiembre de 2016.

\item  Consejera Superior Titular en la UNLP por el Claustro de
Profesores de la Facultad de Cs. Astron\'omicas y Geof\'\i sicas per\'\i odo 
mayo de 2018 - mayo de 2022. Miembor de la Comisi\'on de Ciencia y T\'ecnica de la UNLP en el mismo per\'\i odo.

\item Miembro de la Comisi\'on Asesora T\'ecnica para evaluaci\ 'on de Becas Doctorales de la UNLP, Comisi\'on Asesora de Exactas. Marzo de 2018.
\end{itemize}
\bigskip

\noindent {\bf 08.2 - En Instituciones Acad\'emicas y Cient\'\i ficas}
\medskip

Secretaria del Subcomit\'e de Sismolog\'\i a y F\'\i sica del Interior
Terrestre del Comit\'e Nacional  de la Uni\'on Geod\'esica  y Geof\'\i
sica Internacional (CNUGGI), desde marzo de 1997 a marzo de 2005.
Tesorera desde 20/04/2017.

\bigskip
\noindent {\bf 08.3 - En la funci\'on p\'ublica no universitaria}

\bigskip
\noindent {\bf 08.4 Profesionales}
\medskip

Geof\'\i sico Junior, en la consultora ``Calcagno y Asoc., Gas Petroleo
y Energ\'\i a'', Sarmiento 765, Capital Federal durante el per\'\i
odo comprendido entre el 22/06/88 hasta el 31/01/89.

\bigskip
\noindent {\bf 9 - Miembro de Jurados (Tesis-Concursos-Otros):\/}
\bigskip

%concursos
\noi{\bf Concursos Docentes}

\medskip

1. Miembro titular de la Comisi\'on Asesora por el claustro de graduados para el 
concurso de un cargo de Profesor Titular de la c\'atedra de "Electr\'onica II", 
realizado durante 1991 en la  Facultad de 
Ciencias Astron\'omicas y Geof\'\i sicas, U.N.L.P.

\sm

2. Miembro titular de la Comisi\'on Asesora por el claustro de graduados para un
concurso de auxiliares docentes  de la c\'atedra de "Sismolog\'\i a",
 realizado durante 1993 en la  Facultad de Ciencias Astron\'omicas y
 Geof\'\i sicas, U.N.L.P.

\sm

3. Miembro titular de la Comisi\'on Asesora por el claustro de graduados para un 
concurso de Jefe de Trabajos Pr\'acticos D.S. en la c\'atedra de "M\'etodos 
S\'\i smicos de Prospecci\'on",  realizado en 
Noviembre de 1998, en la  Facultad de Ciencias Astron\'omicas y Geof\'\i sicas, 
U.N.L.P.

\sm

4. Presidente de la Comisi\'on Asesora (claustro de Profesores) para el concurso 
de Ayudante Diplomado D.S. para la c\'atedra de "M\'etodos Potenciales de
 Prospecci\'on", durante 1999, en la  
Facultad de Ciencias Astron\'omicas y Geof\'\i sicas, U.N.L.P.

\sm
5. Presidente de la Comisi\'on Asesora (claustro de Profesores) para  el concurso 
de Jefe de Trabajos Pr\'acticos D.S. para la c\'atedra de "M\'etodos Potenciales de
 Prospecci\'on", durante 2003, en la  
Facultad de Ciencias Astron\'omicas y Geof\'\i sicas, U.N.L.P.

\sm

6. Miembro de la Comisi\'on Asesora por el claustro de Profesores para el concurso 
de Jefe de Trabajos Pr\'acticos D.S.E. para la c\'atedra de "Geomagnetismo y 
Electricidad Atmosf\'erica", durante 2003, en la  Facultad de Ciencias 
Astron\'omicas y Geof\'\i sicas, U.N.L.P.
\sm

7. Miembro de la Comisi\'on Asesora por el claustro de Profesores para los 
concursos de Jefe de Trabajos Pr\'acticos D.E. y Jefe de Tr\'abajos Pr\'acticos 
D. S. para la c\'atedra de "Introducci\'on a la Geof\'\i sica", durante 2003, 
en la  Facultad de Ciencias Astron\'omicas y Geof\'\i sicas, U.N.L.P.

\sm

8. Presidente de la Comisi\'on Asesora (claustro de Profesores) para el concurso 
de Ayudante Diplomado D.S. para la c\'atedra de "M\'etodos Potenciales de 
Prospecci\'on" durante 2004, en la  Facultad de Ciencias Astron\'omicas y
 Geof\'\i sicas, U.N.L.P.
\sm

9. Presidente de la Comisi\'on Asesora (claustro de Profesores) para un cargo de
 Ayudante Diplomado D.S. para la c\'atedra de "M\'etodos S\'\i smicos de 
Prospecci\'on",  durante 2004, en la  Facultad de Ciencias Astronomicas y Geofisicas, U.N.L.P.
\sm

10. Presidente de la Comisi\'on Asesora (claustro de Profesores) del concurso para cubrir 6
 cargos de  Ayudante Alumno  para la c\'atedra de "An\'alisis Matem\'atico I" 
(Expte. 1100-1811/07),  durante Octubre 2007, en la  Facultad de Ciencias Astron\'omicas y 
Geof\'\i sicas, U.N.L.P.
\sm

11. Presidente de la Comisi\'on Asesora (claustro de Profesores) del concurso para cubrir 
6 cargos de  Ayudante Alumno para la c\'atedra de "\'Algebra" (Expte. 1100-1810/07), 
durante Octubre 2007, en la  Facultad de Ciencias Astron\'omicas y Geof\'\i sicas, U.N.L.P.
\sm

12. Miembro de la Comisi\'on Asesora por el claustro de Profesores para el
concurso para cubrir  un cargo de Ayudante Alumno para la c\'atedra de "Ecuaciones Diferenciales 
Parciales", durante marzo de 2008, 
en la  Facultad de Ciencias Astron\'omicas y Geof\'\i sicas, U.N.L.P.

\sm
13. Miembro de la Comisi\'on Asesora por el claustro de Profesores para el
concurso para cubrir un cargo de Jefe de Trabajos Pr\'acticos para la c\'atedra de 
``Geodesia'', durante julio 2008, 
en la  Facultad de Ciencias Astron\'omicas y Geof\'\i sicas, U.N.L.P.

\sm
14. Presidente de la Comisi\'on Asesora por el claustro de Profesores para el
concurso para cubrir un cargo de Jefe de Trabajos Pr\'acticos y un cargo de 
Ayudante Diplomado D.S. para la c\'atedra de ``M\'etodos Potenciales de 
Prospecci\'on'', durante marzo de 2009, 
en la  Facultad de Ciencias Astron\'omicas y Geof\'\i sicas, U.N.L.P.
\sm

15. Miembro de la Comisi\'on Asesora por el claustro de Profesores para el
concurso para cubrir un cargo de Jefe de Trabajos Pr\'acticos D.E. para la 
c\'atedra de ``Geomagnetismo'', durante noviembre de 2009, FCAG -U.N.L.P.
\sm

16. Miembro de la Comisi\'on Asesora por el claustro de Profesores para el
concurso para cubrir un cargo de Ayudante Diplomado D.S. para la c\'atedra de 
``Gravimetr\'\i a'', durante noviembre de 2009, FCAG-UNLP.
\sm

17. Miembro de la Comisi\'on Asesora por el claustro de Profesores para el
concurso para cubrir un cargo de Ayudante Alumno para la c\'atedra de 
``Geodesia'', durante noviembre de 2009, 
en la  Facultad de Ciencias Astron\'omicas y Geof\'\i sicas, U.N.L.P.

\sm

18. Miembro titular de la Comisi\'on Asesora por el claustro de Profesores para un 
concurso de Profesor Asociado D.E. en la c\'atedra de ''M\'etodos S\'\i smicos de
 Prospecci\'on", realizado en junio de 2010.
\sm

19. Miembro titular de la Comisi\'on Asesora por el claustro de Profesores 
para un concurso de Profesor Adjunto D.E. en la c\'atedra de "F\'\i sica del 
Interior Terrestre", realizado en junio de 2010.
\sm

20. Integrante de la Comisi\'on Asesora para el otorgamiento de ayudas para la
realizaci\'on de doctorados y movilidad de doctorados, en Octubre de 2010 y 
Noviembre de 2011.

\sm

21. Miembro titular de la Comisi\'on Asesora por el claustro de Profesores 
para un concurso para cubrir un cargo de Ayudante Diplomado para la c\'atedra de 
``Gravimetr\'\i a'', durante septiembre de 2011, 
en la  Facultad de Ciencias Astron\'omicas y Geof\'\i sicas, U.N.L.P.

\sm
22. Miembro titular de la Comisi\'on Asesora por el claustro de Profesores 
para cubrir interinamente un cargo de Profesor Adjunto con DS para la c\'atedra de 
``Mec\'anica del Continuo'', durante diciembre de 2011, 
en la  Facultad de Ciencias Astron\'omicas y Geof\'\i sicas, U.N.L.P.

\sm

23. Miembro titular de la Comisi\'on Asesora por el claustro de Profesores 
para el otorgamiento de pr\'orrogas de 4 cargos de auxiliares docentes en 
la  Facultad de Ciencias Astron\'omicas y Geof\'\i sicas, U.N.L.P. en marzo de 2013.
% ay dipl. ds fisica de la atmosfera, metodos pot, JTP ds met pot., Instrum. Electr
% para el Observ. de Trelew.

\sm
24. Miembro titular de la Comisi\'on Asesora por el claustro de Profesores 
para la designaci\'on interina de un cargo de Jefe de Trabajos Pr\'acticos con DS 
para Geomagnetismo y Aeronom\'\i a en 
la  Facultad de Ciencias Astron\'omicas y Geof\'\i sicas, U.N.L.P. en Agosto de 2013.
% registro de aspirantes
\sm

25. Miembro titular de la Comisi\'on Asesora por el claustro de Profesores para
el llamado a concurso de un cargo de Ayudante Diplomado Ordinario con DS en 
la c\'atedra de M\'etodos El\'ectricos de Prospecci\'on, en Noviembre de 2013.
\sm

26. Miembro titular de la Comisi\'on Asesora por el claustro de Profesores para
la designaci\'on interina de un cargo de Jefe de Trabajos Pr\'acticos con DS
en la c\'atedra de Geof\'\i sica General, en Abril de 2014.
% registro de aspirantes
\sm

27. Miembro titular de la Comisi\'on Asesora por el claustro de Profesores 
para la designaci\'on interina de un cargo de Ayudante Diplomado DS para 
Geomagnetismo y Aeronom\'\i a en 
la  Facultad de Ciencias Astron\'omicas y Geof\'\i sicas, U.N.L.P. en agosto de 2014.
% registro de aspirantes

\sm

28. Presidente de la Comisi\'on Asesora por el claustro de Profesores para los
concursos para cubrir un cargo de Jefe de Trabajos Pr\'acticos en marzo de 2017 y un cargo de 
Ayudante Diplomado D.S. en octubre de 2017, para la c\'atedra de ``M\'etodos Potenciales de 
Prospecci\'on'', 
en la  Facultad de Ciencias Astron\'omicas y Geof\'\i sicas, U.N.L.P.
\medskip

\noi {\bf Tesis Doctorales y de Magister}

%\sm
%\noi Integrante de una mesa examinadora del Seminario de Postgrado 
%``Sismometr\'\i a e Instrumentaci\'on Digital'' en agosto de 1998. 

\sm
Integrante Titular del Tribunal de la Tesis Doctoral del Geof. Luis Guarracino en el tema:
"Modelado num\'erico del flujo de aguas subterr\'aneas y transporte de solutos en
 medios porosos heterog\'eneos", Expte. DAP 15/00, resol. HCA 27/03/01.
Fecha de defensa 05/09/2001, Facultad de Ciencias Astron\'omicas y Geof\'\i sicas, 
U.N.L.P.
\sm

Integrante Titular del Tribunal de la Tesis Doctoral de la Geof. Gabriela A. Badi en el tema:
"Atenuaci\'ion s\'\i smica en la regi\'on de Nuevo Cuyo", Fecha de defensa 12 de Julio de 2011, en 
la Facultad de Ciencias Astron\'omicas y Geof\'\i sicas, U.N.L.P.

\sm
Integrante Titular del Tribunal de la Tesis Doctoral del Geof. Juan Ignacio Sabbione en el tema:
"Algoritmos matem\'aticos y computacionales para la detecci\'on autom\'atica de se\~nales 
s\'\i smicas", Fecha de defensa 26 de Marzo de 2012, en 
la Facultad de Ciencias Astron\'omicas y Geof\'\i sicas, U.N.L.P.
\sm

Integrante Titular del Tribunal de la Tesis de Maestr�a en Geom\'atica del 
Ing. Johnny Alexander Vega Gutierrez en el tema: "Estimaci\'on del riesgo por deslizamientos de laderas generados por eventos s\'\i smicos en la ciudad de Medell\'\i n usando herramientas de la Geom\'atica",  Fecha de defensa 22  de febrero 2013, en 
la Facultad de Ciencias Astron\'omicas y Geof\'\i sicas, U.N.L.P.

\sm
Integrante Titular del Tribunal de la Tesis Doctoral de la Geof. Ana Carolina  Pedraza de Marchi en el tema: "Caracterizaci\'on isost\'atica del sector 
volc\'anico del margen continental argentino". 
Fecha de defensa 25 de Marzo de 2015, en 
la Facultad de Ciencias Astron\'omicas y Geof\'\i sicas, U.N.L.P.
\sm

Presidente del Tribunal de la Tesis Doctoral de la Geof. Laura Mariana Longo en el tema: \emph{Caracterizaci\'on de la estructura del complejo volc\'anico Auca Mahuida mediante datos aeromagn\'eticos y gravim\'etricos}, designada en Octubre de 2016.
Fecha de Defensa 12/06/2017, en 
la Facultad de Ciencias Astron\'omicas y Geof\'\i sicas, U.N.L.P.

\sm

Presidente del Tribunal de la Tesis Doctoral de la Geof. Mar\'\i a Laura G\'omez Dacal 
sobre el tema: \emph{Caracterizaci\'on del Macizo Norpatag\'onico por medio del 
an\'alisis integrado de informaci\'on geof\'\i sica y geol\'ogica}. Fecha de defensa 6 de julio de 2017, en 
la Facultad de Ciencias Astron\'omicas y Geof\'\i sicas, U.N.L.P.

\sm

Presidente del Tribunal de la Tesis Doctoral del Ing. Robiel Mart\'\i nez Corredor, sobre el tema: \emph{Simulaci\'on num\'erica de propagaci\'on de ondas en reservorios de hidrocarburos fracturados utilizando el m\'etodo de elementos finitos}. 
 Fecha de Defensa 18/12/2017,  en la Facultad de Ingenier\'\i a UNLP.
\sm

Presidente del Tribunal de la Tesis Doctoral del Geof. Santiago Solazzi
sobre el tema: \emph{Modelado y an\'alisis de la respuesta s\'\i smica de rocas heterog\'eneas saturadas por fluidos inmiscibles}. Fecha de defensa 18 de mayo de 2018, en 
la Facultad de Ciencias Astron\'omicas y Geof\'\i sicas, U.N.L.P.

\medskip

\noi{\bf Tesis de Grado}
\sm

\noi Miembro del jurado de una Tesis de Grado de la carrera de Geof\'\i sica
(Geof. Juan Carlos Soldo), en Marzo de 1998.
\sm

\noi Miembro del jurado de dos Tesis de Grado de la carrera de Geof\'\i sica
(Geof. Adolfo Lugones y Geof. Valeria Grosfeld), en Diciembre de 1998.
\sm

\noi Miembro del jurado de dos Tesis de Grado de la carrera de Geof\'\i sica 
en Noviembre de 2003 (Geof. Andrea L. Ag\"uin y Geof. Gabriela Zanca).
\sm

\noi Miembro del jurado de una Tesis de Grado de la carrera de Geof\'\i sica
(Geof. Leonardo Monachessi), en Septiembre de 2006.

\sm

\noi Miembro del jurado de una Tesis de Grado de la carrera de Geof\'\i sica
(Geof. Sebasti\'an Blanco), en Noviembre de 2006.

\sm

\noi Miembro del jurado de una Tesis de Grado de la carrera de Geof\'\i sica
(Geof. Pablo F. Alvarez), en Agosto de 2007.

\sm
\noi Miembro del jurado de una Tesis de Grado de la carrera de Geof\'\i sica
(Geof. Ana Carolina Pedraza De Marchi), en Diciembre de 2009.
\sm

\noi Miembro del jurado de una Tesis de Grado de la carrera de Geof\'\i sica
(Geof. Ma. Laura Gomez Dacal), en Marzo de 2012.
%28/03/12
\sm

\noi Miembro del jurado de la Tesis de Grado de la carrera de Geof\'\i sica
del Geof. Guido Panizza, en Diciembre de 2013.
%20/12/2013
\sm

\noi Miembro del jurado de la Tesis de Grado de la carrera de Geof\'\i sica
de la Geof. Soledad Lagos, en Marzo de 2014.
%28/03/2014

\sm
%2015
\noi Miembro del jurado de la Tesis de Grado de la carrera de Geof\'\i sica
del Geof. Gabriel Gelpi, en Marzo de 2015.
%30/03/2015

\sm

\noi Miembro del jurado de la Tesis de Grado de la carrera de Geof\'\i sica
del Geof. Ariel Sanchez Camus, en Abril de 2015.
%21/04/2015
\sm

\noi Miembro del jurado de la Tesis de Grado de la carrera de Geof\'\i sica
del Geof. Jhonatan Pendiuk, en Febrero de 2016.
%26/02/2016

\bigskip

%referatos
\noindent {\bf Referato de Art\'\i culos}

\ms
\noi  Referato de 2 art\'\i culos para el Journal of Computational Acoustics 
en 1999.
\sm

\noi  Referato de 10 art\'\i culos para la 64th. Conference and  Exhibition,
European Association of Geoscientists and Engineers, Florencia, Italia, 27
-30 de Mayo de 2002. 
\sm

\noi Referato de 8 art\'\i culos para la 65th. Conference and  Exhibition,
European Association of Geoscientists and Engineers, Stavanger, Noruega, 2 -
5 de Junio  de 2003. 
\sm

\noi  Referato de 3 art\'\i culos para la 66th. Conference and  Exhibition,
European Association of Geoscientists and Engineers, Paris, Francia, 7 - 10
de Junio  de 2004. 
\sm

\noi  Referato de 1 art\'\i culo para Mec\'anica Computacional, Vol. 23, en 
Septiembre de 2004.
\sm

\noi  Referato de 1 art\'\i culo para el Journal of Geophysics and Engineering  
en Octubre de 2004.
\sm

\noi  Referato de 2 art\'\i culos para las Actas de la
XXII Reuni\'on  Cient\'\i  fica  de  la  Asociaci\'on
Argentina de Geof\'\i sicos y Geodestas, Buenos Aires, 6 al 10 de Septiembre
 de 2004.
\sm

\noi  Referato de 5 art\'\i culos para la 67th. Conference and  Exhibition,
European Association of Geoscientists and Engineers, Madrid, Espa\~na, 13 - 16
de Junio  de 2005.
\sm

\noi  Referato de 1 art\'\i culo para el International Journal of Solids
and Structures, en Noviembre de 2005.
\sm

\noi  Referato de 5 art\'\i culos para la 69th. Conference and  Exhibition,
European Association of Geoscientists and Engineers (London 2007), Londres, 11 - 14 de Junio  de 2007.

\sm
\noi  Referato de 1 art\'\i culo para el Bolet\'\i n del Instituto de 
Fisiograf\'\i a y Geolog\'\i a, Universidad Nacional de Rosario, en 
Agosto de 2007.
\sm

\noi  Referato de 4 art\'\i culos para la 70th. Conference and  Exhibition,
European Association of Geoscientists and Engineers (Rome 2008), Roma, 9 - 12
de Junio  de 2008.
\sm

\noi  Referato de 6 art\'\i culos para la 71th. Conference and  Exhibition,
European Association of Geoscientists and Engineers (Amsterdam 2009), Amsterdam, 
8 de Junio de 2009.

\sm
\noi  Referato de 9 art\'\i culos para la 72th. Conference and  Exhibition,
European Association of Geoscientists and Engineers (Barcelona 2010), Bacelona  
14 - 17 de Junio de 2010.

\sm
\noi  Referato de 12 art\'\i culos para la 73th. Conference and  Exhibition,
European Association of Geoscientists and Engineers (Vienna 2011), Viena, Austria,   
23-26 de Mayo de 2011.
\sm

\noi  Referato de 1 art\'\i culo para el Journal of Applied Geophysics, en 
Abril de 2011.

\sm
\noi  Referato de 10 art\'\i culos para la 74th. Conference and  Exhibition,
European Association of Geoscientists and Engineers \& SPE EUROPEC 2012 (Copenhagen 2012), Dinamarca,  4-7 de Junio de 2012.
\sm

\noi  Referato de 11 art\'\i culos para la 75th. Conference and  Exhibition,
European Association of Geoscientists and Engineers \& SPE EUROPEC 2013 (London 2013), Londres, 10-13 de Junio de 2013.
\sm

\noi  Referato de 8 art\'\i culos para la 76th. Conference and  Exhibition,
European Association of Geoscientists and Engineers (Amsterdam 2014), Amsterdam, 
16-19 de Junio de 2014.
\sm

\noi  Referato de 1 art\'\i culo para Materials Research Innovations (Maney Publishing), en Septiembre de 2014.
\sm

\noi  Referato de 1 art\'\i culo para la revista Ciencia, Tecnolog\'\i a y Futuro, editada por la Empresa Nacional de Petr\'oleos de Colombia,
ECOPETROL, en Diciembre de 2014.
\sm

\noi  Referato de 8 art\'\i culos para la 77th. Conference and  Exhibition,
European Association of Geoscientists and Engineers (Madrid 2015),
Earth Science for Energy and Environment, Madrid, 1 -4 de Junio de 2015.
\sm

\noi  Referato de 5 art\'\i culos para la 78th. Conference and  Exhibition,
European Association of Geoscientists and Engineers (Vienna 2016),
Earth Science for Energy and Environment, Viena 30 de Mayo-2 de Junio de 2016. 

\sm

\noi  Referato de 1 art\'\i culo para la revista Ciencia, Tecnolog\'\i a y Futuro, 
editada 
por la Empresa Nacional de Petr\'oleos de Colombia,
ECOPETROL, en Marzo de 2016.
\sm

\noi  Referato de 1 art\'\i culo para la revista Journal of Porous Media, 
en Diciembre de 2016.
\sm

\noi  Referato de 6 art\'\i culos para la 79th. Conference and  Exhibition,
European Association of Geoscientists and Engineers (Paris 2017),
Earth Science for Energy and Environment, Paris 12-15 de Junio de 2017.
%para EAGE 2018 no me enviaron trabajos a evaluar
\sm

\noi  Referato de 2 art\'\i culos para las Actas de la
XXVIII Reuni\'on  Cient\'\i  fica  de  la  Asociaci\'on
Argentina de Geof\'\i sicos y Geodestas, La Plata, 17 al 21 de Abril  de 2017.
%\sm

%\noi  Referato de 1 art\'\i culo para la revista Geofluids en Mayo de 2018.
\bs

\noindent {\bf 10 - Carrera de Investigador (CIC, CONICET, OTROS):\/}
\bigskip

Miembro de la Carrera del Investigador
Cient\'\i fico y Tecnol\'ogico del CONICET.  (\'area Ciencias Exactas y Naturales, Ciencias de la Tierra, el Agua y la Atm\'osfera):\sm

\noi 1. en la categor\'\i a de Investigador Asistente, por resoluci\'on 1281 del 23/08/02 (fecha de ingreso 01/12/02).\sm

\noi 2. Promoci\'on a la categor\'\i a de Investigador Adjunto por Resol. 2132 del
 15/12/05, efectiva a partir del 01/01/2006.\sm
 
\noi 3. Promoci\'on a la categor\'\i a de {\bf Investigador Independiente} por Resol. 1828 del
 21/06/2011, efectiva a partir del 01/06/2011. \medskip

Lugar de trabajo: Depto. de  Geof\'\i sica Aplicada, Facultad de Ciencias 
Astron\'omicas y Geof\'\i sicas, U.N.L.P.
\bigskip

\noindent {\bf 11 - Subsidios Recibidos:\/}
\bigskip

A continuaci\'on se detallan diversas fuentes de las que he recibido apoyo
 econ\'omico, debiendo aclararse que cuando se trata de
subsidios recibidos por proyecto, s\'olo debe computarse una fracci\'on de los
mismos.

\medskip

\noi Instituci\'on otorgante: Universidad Nacional de La Plata. Subsidios 
autom\'aticos por Programa de Incentivos 1995-2007.
%hay detalles en info de mayor ded. 2006.
Titular: Dr. Juan E. Santos.

\sm
\noi Instituci\'on otorgante: Facultad de Ciencias Astron\'omicas y Geof\'\i 
sicas, para participar en el IV Congreso Internacional de Ciencias de la
Tierra, realizado en Santiago, Chile. Per\'\i  odo: del 5 al 9 de agosto de
1996.  Monto: \$300.

\sm
\noi Instituci\'on otorgante: Facultad de Ciencias Astron\'omicas y Geof\'\i 
sicas, para participar en la XIX Reuni\'on Cient\'\i  fica de la A.A.G.G.,
realizada en San Juan: del 28 al 31 de octubre de 1997.  Monto: \$200.

\sm
\noi Instituci\'on otorgante: Uni\'on Europea, Proyecto Joule, por 
participaci\'on en el proyecto: ``Detection of Overpressured Zones from Seismic and Well
Data''. Monto asignado para 1998: US\$ 9182. Monto asignado para 1999: US\$
8774.

\sm
\noi Instituci\'on otorgante: Universidad Nacional de La Plata. Subsidios para
viajes y estad\'\i  as, Secretar\'\i  a de Ciencia y T\'ecnica, para
participar en la Primer Reuni\'on T\'ecnica del Subcomit\'e de Sismolog\'\i  a
y F\'\i  sica del Interior Terrestre, San Juan, del 18 al 19 de junio de 1998.
Monto: \$300.

\sm
\noi Instituci\'on otorgante: Asociaci\'on Argentina de Mec\'anica
Computacional. Beca para participar del IV World Congress on Computational
Mechanics, Buenos Aires, 29 de junio al 2 de julio de 1998. Monto: \$400.

\sm
\noi Instituci\'on otorgante: Agencia Nacional de Promoci\'on Cient\'\i  fica y
Tecnol\'ogica (programa FONCyT) PICT97, por el proyecto ''Propagaci\'on de ondas y
flujo, transporte de contaminantes en el subsuelo terrestre'', Titular: Dr.
Juan E. Santos. 

\sm
\noi Instituci\'on otorgante: CONICET, Proyecto de Investigaci\'on Plurianual
(PIP) Nro. 0363/98, Titular: Dr. Juan E. Santos. Monto asignado: tres cuotas 
de \$6.089.

\sm
\noi Instituci\'on otorgante: CONICET, Subsidio por ingreso a Carrera del
Investigador Resol. 1581, 10/09/04. Titular: Claudia L. Ravazzoli. 
Monto asignado: \$5000.

\sm

\noi Instituci\'on otorgante: Agencia Nacional de Promoci\'on Cient\'\i  fica y
Tecnol\'ogica, PICT 03-13376, duraci\'on 3 a\~nos. Monto para el primer 
a\~no \$ 59146 (a partir de junio 2005). Responsable: Dr. Juan E. Santos.

\sm
\noi  Instituci\'on otorgante: CONICET, PIP 04-5126, duraci\'on 2 a\~nos. 
Monto para el primer a\~no \$ 48000 (a partir de octubre de 2005).
Responsable: Dr. Juan E. Santos.

\sm

\noi Instituci\'on otorgante: Universidad Nacional de La Plata. Subsidio
Autom\'aticos por Programa de Incentivos, Proyecto 11G097. Per\'\i odo 01/01/2008 - 31/12/2011.
Monto total \$ 40100. Responsable: Dr. Juan E. Santos.

\sm

\noi  Instituci\'on otorgante: CONICET, PIP 112-200801-00952,  
``M\'etodos num\'ericos para la resoluci\'on de problemas en Geof\'\i sica
Aplicada''. Monto total \$ 272000 (a partir de 2009).
Responsable: Dr. Juan E. Santos.

\sm
\noi  Instituci\'on otorgante: CONICET, PIP 112-201101-00777,  
``Simulaci\'on num\'erica en medios porosos saturados''. Monto total \$ 135000 (PIP 2012-2014).
Responsable: Dr. Juan E. Santos.

\sm

\noi  Instituci\'on otorgante: Universidad Nacional de La Plata,  
Subsidio para adquisici\'on o mejora de equipamientos 2011''. Monto total \$ 8000,
fecha de aprobacion Noviembre 2011.
Responsable: Dra. Claudia Ravazzoli
%fecha de compra 2 PCs + monitor: mayo de 2012

\sm

\noi  Instituci\'on otorgante: CONICET, PIP 112-201501-00192,  
 (PIP 2015-2017).
"Modelado te\'orico y computacional en monitoreo y exploraci�n geof\'\i sica",
Monto total \$ 145000. 
Responsable: Dra. Claudia Ravazzoli, Dr. Fabio Zyserman.

\sm

\noi Instituci\'on otorgante: Universidad Nacional de La Plata. Subsidio
Autom\'aticos por Programa de Incentivos, Proyecto 11G145. 
"Modelado te\'orico y computacional de monitoreo y exploraci\'on en geof\'\i sica 
aplicada".
Per\'\i odo 01/01/2016 - 31/12/2019.
Monto total \$ 90000 (estimado). Responsable: Dr. Fabio Zyserman.

\bs

\noindent {\bf 12 - Sociedades Acad\'emicas y Profesionales
de las cuales es miembro:\/}
\bigskip


\noi Miembro Asociado de la Society of Exploration Geophysicists (SEG).

\sm

\noi Miembro Asociado de la European Association of Geoscientists and Engineers
(EAGE).

\sm

\noi Miembro del Subcomit\'e  de Sismolog\'\i a  y F\'\i sica  del Interior
Terrestre del Comit\'e Nacional  de la Uni\'on Geod\'esica  y Geof\'\i
sica Internacional (CNUGGI).

\sm

\noi Miembro de la American Geophysical Union (AGU) en 2012.

%\sm
%\noi Miembro  Activo  de  la  Asociaci\'on  Argentina  de  Geof\'\i sicos y
%Geodestas.
%\sm
%\noi Miembro  Activo  de  la  Asociaci\'on  Argentina  de  Mec\'anica
%Computacional (AMCA).
\sm


\bigskip
\noindent {\bf 13 - Patentes - Convenios:\/}
\bigskip

\noi Participante del convenio: ``Simulaci\'on
num\'erica aplicada a problemas de propagaci\'on de ondas", Programa
de Cooperaci\'on Cient\'\i fica entre la Rep\'ublica
Argentina y E.E.U.U.

\noi Per\'\i odo: 1992--1994.

\noindent Investigador extranjero: Jim~Douglas, Jr.

\noi Aprobado por Resoluci\'on  del Directorio de CONICET Nro. 1008 de fecha
30/6/1992.

\medskip

\noi Participante del convenio: ``Simulaci\'on
num\'erica de fen\'omenos de  propagaci\'on de
ondas con aplicaci\'on a modelado e inversi\'on en Geof\'\i sica",
  Programa
de Cooperaci\'on Cient\'\i fica entre la Rep\'ublica
Argentina y E.E.U.U.

\noindent Per\'\i odo : 1996--1998.

\noindent Investigador extranjero: Jim~Douglas, Jr.

\noi Aprobado por Resoluci\'on  del Directorio de CONICET Nro. 0583 de fecha 24
de abril de 1996.

\ms

 \noi Participante del convenio: ''Detection of Overpressured Zones from
Seismic and Well Data'', en el marco del Proyecto Joule,  Uni\'on Europea
(contrato DG 12 JOF3-CT97-0036), en car\'acter de Asistencia Externa al
Osservatorio Geofisico Sperimentale, Trieste (Italia). 

\noi Per\'\i odo: 01/12/1997-- 01/12/2000.

\noi Investigador extranjero: Jos\'e M. Carcione.

\ms

\noi Integrante grupo responsable del proyecto: ``CO$_2$ Geological Storage: research into monitoring 
and verification technology CO2REMOVE'', financiado en el marco del 6th. Framework 
Programme Sustainable Energy Systems, European Community and The Netherlands 
Organization (TNO),
contrato 518350(SES6), desde el 01/09/2007 al 31/03/2011. Monto total 90100 Euros.

\bigskip

\noindent {\bf 14 - Seminarios - Conferencias y Cursos Dictados:\/}
\bigskip
\noindent A\~no: 1992

\noindent Lugar: Comisi\'on Nacional de Energ\'\i a At\'omica, Centro
At\'omico Constituyentes.

\noindent Tema: {\it Propagaci\'on de Ondas en Medios Viscoel\'asticos}

\medskip
\noindent A\~no: 2011 (07/09/11)

\noindent Lugar: Facultad de Ciencias Astron\'omicas y Geof\'\i sicas, Taller de Transferencia.

\noindent Tema: {\it Monitoreo sismico de repositorios de CO2 en el marco del proyecto CO2REMOVE.}

\medskip
\noindent A\~no: 2012 (12/09/12)

\noindent Lugar: Facultad de Ciencias Astron\'omicas y Geof\'\i sicas, Ciclo de Coloquios.

\noindent Tema: {\it Modelos de reflectividad sismica en repositorios de CO$_2$}

\medskip
\noindent A\~no: 2012 (08/06/12)

\noindent Lugar: Fac. de Ciencias Astron\'omicas y Geof\'\i sicas, Bolet\'\i n de Noticias del Observatorio. 

\noindent Tema: {\it Donde hubo petr\'oleo, CO2 se puede ubicar. Una historia de almacenamiento y monitoreo permanente}.  Entrevista publicada en 
{\it http://www.fcaglp.unlp.edu.ar/extension y difusion/boletin/324/noticias}


\bigskip

\noindent {\bf 15 - Participaci\'on en Congresos, Encuentros, Jornadas
y Simposios}

\bigskip
\noi Evento:  {\it 14ta.  Reuni\'on  Cient\'\i  fica  de  la  Asociaci\'on
Argentina de Geof\'\i sicos y Geodestas}

\noi Car\'acter: asistente \quad Lugar: Mendoza \quad
Fecha: Octubre de 1986.
\ms
\noi Evento:  {\it 16ta.  Reuni\'on  Cient\'\i  fica  de  la  Asociaci\'on
Argentina de Geof\'\i sicos y Geodestas}

\noi Car\'acter: expositora \quad Lugar: Bah\'\i a Blanca\qquad
Fecha: 22 al 26 de Octubre de 1990.
\ms
\noi Evento:  {\it IV  Reuni\'on   de  Trabajo  en  Procesamiento   de  la
Informaci\'on y Control}

\noi Car\'acter:  co-autora \quad  Lugar:  Buenos Aires \quad
Fecha: del 18 al 22 de Noviembre de 1991.
\ms
\noi Evento: {\it 17ma.  Reuni\'on Cient\'\i fica de la
Asociaci\'on Argentina de Geof\'\i sicos y Geodestas}

\noi Car\'acter:  expositora \quad  Lugar:  Buenos Aires
\quad Fecha: del 26 al 30 de Octubre de 1992.
\ms
\noi Evento: {\it 18va.  Reuni\'on Cient\'\i fica de la
Asociaci\'on Argentina de Geof\'\i sicos y Geodestas}

\noi Car\'acter:  expositora \quad  Lugar:  La Plata
\quad Fecha: del 24 al 28 de Octubre de 1994.
\ms
\noi Evento: {\it IV Congreso Internacional de Ciencias
de la Tierra}

\noi Car\'acter:  expositora \quad  Lugar: Santiago, Chile
\quad Fecha: del 5 al 9 de Agosto de 1996.
\ms
\noi Evento: Scientific Assembly of the
International Association of Geodesy, IAG 97

\noi Car\'acter: co-autora \quad  Lugar: R\'\i o de Janeiro,
Brasil \quad Fecha: del 3 al 9 de Septiembre de 1997.
\ms
\noi Evento: {\it 19na.  Reuni\'on Cient\'\i fica de la
Asociaci\'on Argentina de Geof\'\i sicos y Geodestas}

\noi Car\'acter:  expositora \quad  Lugar:  San Juan
\quad Fecha: del 28 al 31 de Octubre de 1997.

\ms
\noi Evento: {\it IV World Congress on Computational Mechanics}, 

\noi Car\'acter: co-autora  \quad  Lugar: Buenos Aires
\quad Fecha: 29 de junio al 2 de julio de 1998.

\ms
\noi Evento: {\it IV Congreso Latinoamericano de Hidrolog\'\i a Subterr\'anea},

\noi Car\'acter: co-autora  \quad  Lugar: Montevideo
\quad Fecha: 16 al 20 de noviembre de 1998.

\ms
\noi Evento: {\it IV International Conference on Theoretical and Computational
Acoustics}

\noi Car\'acter:  expositora \quad  Lugar: Trieste, Italia
\quad Fecha: del 10 al 14 de Mayo de 1999.

\ms
\noi Evento: {\it 62nd. European Association of Geoscientists and Engineers
Conference and Technical Exhibition}

\noi Car\'acter:  expositora \quad  Lugar: Glasgow, Escocia
\quad Fecha: del 29 de Mayo al 2 de Junio de 2000.
\ms

\noi Evento: {\it Society
of Exploration Geophysicists, International Exposition and 71st. Annual
Meeting}

\noi Car\'acter:  co-autora \quad  Lugar: San Antonio, Texas, EEUU.
\quad Fecha: 9 al 14 de septiembre de 2001.

\ms
\noi Evento: {\it 64th. European Association of Geoscientists and Engineers
Conference and Technical Exhibition}

\noi Car\'acter: co-autora \quad  Lugar: Florencia, Italia
\quad Fecha: del 27 al 30 de Mayo de 2002.

\ms
\noi Evento:
{\it 7mo. Congreso Argentino de Mec\'anica Computacional (MECOM 2002)}. 

\noi Car\'acter: co-autora \quad  Lugar: Paran\'a, Santa Fe.
\quad Fecha: 28 al 31 de Octubre de 2002.

\ms
\noi Evento:
{\it 14to. Congreso sobre M\'etodos Num\'ericos y sus Aplicaciones (ENIEF 2004)}. 

\noi Car\'acter: co-autora \quad  Lugar: Bariloche.
\quad Fecha: 8 al 11 de Noviembre de 2004.
\ms

\noi Evento:
{\it 8vo. Congreso Argentino de Mec\'anica Computacional (MECOM 2005}. 

\noi Car\'acter: expositora de un trabajo y co-autora de un poster.\newline
 Lugar: Buenos Aires.
\quad Fecha: 16 al 18 de Noviembre de 2005.
\ms


\noi Evento: {\it Society of Exploration Geophysicists, International Exposition 
and 76th. Annual Meeting}

\noi Car\'acter:  co-autora \quad  Lugar: New Orleans, Louisiana, EEUU.
\quad Fecha: 1 al 6 de octubre de 2006.

\ms

\noi Evento: {\it EAGE/EAGO/SEG International Conference and Exhibition} 

\noi Car\'acter:  co-autora \quad  Lugar: Saint Petersburg 2006,  Russia.\quad 
Fecha: 16 - 19 de Octubre de 2006.
\ms

\noi Evento: {\it Society of Exploration Geophysicists, International Exposition 
and 77th. Annual Meeting}

\noi Car\'acter:  co-autora \quad  Lugar: San Antonio, Texas, EEUU.
\quad Fecha: 23 al 28 de Septiembre de 2007.

\ms

\noi Evento: {\it XVI Congerso sobre M\'etodos Num\'ericos y sus Aplicaciones - 
I Congreso de Matem\'atica Aplicada, Computacional e Industrial MACI 2007}

\noi Car\'acter:  co-autora \quad  Lugar: Cordoba, Argentina.
\quad Fecha: 2 al 5 de Octubre de 2007.

\ms

\noi Evento: {\it Society of Exploration Geophysicists, International Exposition 
and 78th. Annual Meeting}

\noi Car\'acter:  co-autora \quad  Lugar: Las Vegas, EEUU.
\quad Fecha: 9 al 14 de Noviembre de 2008.

\ms

\noi Evento: {\it 4to. Encuentro Internacional E-ICES 4, International Center for Earth 
Sciences},

\noi Car\'acter:  co-autora \quad  Lugar: Malarg\"ue, Mendoza.
Fecha: 29 al 31 de Octubre 2008.
\ms

\noi Evento: {\it 24ta. Reuni\'on  Cient\'\i  fica  de  la  Asociaci\'on
Argentina de Geof\'\i sicos y Geodestas}

\noi Car\'acter:  co-autora \quad  Lugar: Mendoza, Argentina.
Fecha: 14 al 17 de abril de 2009.
\ms

\noi Evento: {\it Segundo Workshop sobre Inversi\'on y Procesamiento de Se\~nales
en Exploraci�n S\'\i smica IPSES 09, F.C.A.G.}

\noi Car\'acter:  expositora \quad  Lugar: La Plata, Argentina.
Fecha: 14 de agosto de 2009. 

% titulo charla 14/08/09 Seismic reflectivity analysis in Co2 sequestration
% problems
\ms

\noi Evento: {\it Taller de Transferencia F.C.A.G. 2011}

\noi Car\'acter:  expositora \quad  Lugar: La Plata, Argentina.
Fecha: 7 de septiembre de 2011. 
% titulo charla: Monotoreo sismico de repositorios de CO2 en el marco del proyecto 
% co2remove

\ms

\noi Evento: {\it CO2ReMoVe Technical Meeting} 
% titulo charla: Modelling of Seismic Response and Indicators for CO2 Sequestration
% Schlumberger Stavanger Research
\noi Car\'acter:  expositora \quad  Lugar: Stavanger, Noruega.
Fecha: 18 de noviembre de 2009.

\bs

\noi Evento: {\it VIII Congreso de 
Exploraci\'on y Desarrollo de Hidrocarburos 2011, Instituto Argentino del Petr\'oleo y del Gas}
% titulo charla: Modelling of Seismic Response and Indicators for CO2 Sequestration
% Schlumberger Stavanger Research
\noi Car\'acter:  coautora \quad  Lugar: Mar del Plata, Argentina.
Fecha: 8-11 de noviembre de 2011.
 
\bs
\noi Evento: {\it CO2ReMoVe Closing Conference} \quad  Car\'acter: exhibici\'on de poster

\noi Trabajo: AVA and AVF characteristics of a carbon dioxide saturation transition,
by Claudia L. Ravazzoli and Juli\'an L. G\'omez.

\noi  Lugar: IFP Energies Nouvelles, Rueil-Malmaison, Francia, \quad 
Fecha: 29 de Febrero de 2012.
\ms

\noi Evento: {\it Primer Workshop Argentino-Alem\'an CONICET-GFZ "Georesources in Argentina". }
\noi Car\'acter:  asistente \quad  Lugar: CONICET, Buenos Aires.
Fecha: 7-8 de Mayo de 2012. 
% completarENIEF MECOM 2011 2012 ver 2013

\ms

\noi Evento: {\it 10mo. Encuentro Internacional E-ICES 10, International Center for Earth Sciences},

\noi Car\'acter:  co-autora \quad  Lugar: Buenos Aires.
Fecha: 3 al 6 de Noviembre de 2014.
\ms

\noi Evento: {\it VIII Workshop Wavelets y Teor\'\i a de la Informaci\'on,
Facultad de Cs. Exactas e Ingenier\'\i a UNLP,} 

\noi Car\'acter:  co-autora \quad  Lugar: Facultad de Ingenier�a UNLP, La Plata.
Fecha: 9 al 11 de agosto de 2016.

\ms
\noi Evento: {\it II Jornada de Estudiantes de Doctorado JAEDOC II,
FCAG - UNLP,} 

\noi Car\'acter:  asistente \quad  Lugar: FCAG- UNLP, La Plata.
Fecha: 9 de septiembre de 2016.

\ms
\noi Evento: {\it XXVIII Reuni\'on  Cient\'\i  fica  de  la  Asociaci\'on
Argentina de Geof\'\i sicos y Geodestas}, 
\noi Car\'acter:  expositora \quad  Lugar: Universidad Nacional de La Plata, Rectorado.
Fecha: La Plata, 17 al 21 de Abril  de 2017.
\bs

\noindent {\bf 16 - Organizaci\'on de Eventos Cient\'\i ficos -
Visitas de Investigadores.\/}

\bigskip
\noi Miembro del Comit\'e Organizador Local de la 18va. Reuni\'on Cient\'\i
fica de la  Asociaci\'on Argentina de  Geof\'\i sicos y  Geodestas, La
Plata, del 24 al 28 de Octubre de 1994.

\ms

\noi Actuaci\'on como  coordinadora por  parte de  la facultad,
junto con  otros miembros  del Depto.  de Geof.  Aplicada, durante las
visitas realizadas por el Profesor Dr. Jim Douglas Jr., del
Center for Applied Mathematics, Purdue University, durante  Septiembre
de 1991, Junio de 1993 y Noviembre de 1994.
\ms

\noi Actuaci\'on como coordinadora por parte  de la facultad, de la  visita
realizada  por  el  Dr.  Jos\'e  Carcione  del  Osservatorio
Geofisico Sperimentale (Trieste), en  el marco del Programa  PROCYTEXT
(S.E.C.Y.T), durante el mes de Marzo de 1994.
\ms

\noi Organizaci\'on de la �Primer
Reuni\'on T\'ecnica del Subcomit\'e de Sismolog\'\i  a y F\'\i  sica del
Interior Terrestre, realizada en San Juan, del 18 al 19 de junio de 1998.

\ms

\noi Organizaci\'on junto con otros miembros del Depto. de Geof\'\i  sica
Aplicada de la visita y conferencia (24/06/98), del Dr. Jos\'e
Carcione, Osservatorio Geofisico Sperimentale, Trieste, durante el mes de
Junio de 1998.

\ms

\noi Organizaci\'on de la visita y la conferencia del Dr. Andrew Hanyga,
University of Bergen, Noruega, 03/07/98.

\ms

\noi Organizaci\'on de la visita y la conferencia del Dr. Robert Kleinberg,
Schlumberger Doll Research, Connecticut, E.E.U.U., el 23/09/98.

\bs 


%\newpage

\noindent {\bf 17 - Participaci\'on en proyectos acreditados de investigaci\'on 
cient\'\i fica, art\'\i stica o desarrollo tecnol\'ogico}
\medskip

\noindent 1) Miembro del  {\it Programa  de  Formaci\'on
Preferencial  de  Recursos   Humanos}, aprobado   por   la   U.N.L.P.,
denominado: {``Modelado y Simulaci\'on Num\'erica de Fen\'omenos de  Propagac
\'on de Ondas en Geof\'\i sica''}, llevado a cabo en  el
Depto. de Geof\'\i sica Aplicada de esta Facultad desde Nov. de 1991
hasta  Junio de 1993.

\medskip
\noindent 2) Integrante del proyecto:
{``Simulaci\'on num\'erica e inversi\'on en problemas de
Geof\'\i sica''}, llevado a cabo en el marco del Programa de Incentivos,
U.N.L.P. desde el 1/07/93 hasta el  1/07/96.
\medskip

\noindent 3) Integrante del proyecto: {``Modelado e Inversi\'on en Geof\'\i
sica Aplicada''}, en el marco del Programa de Incentivos,
U.N.L.P. desde el 1/04/96 hasta el 30/04/98.
\ms

\noi 4) Integrante del proyecto: "Modelado num\'erico de propagaci\'on de
ondas, flujo de aguas subterr\'aneas, transporte de contaminantes en el
subsuelo y resoluci\'on de problemas inversos asociados", el que se
desarrolla en el Depto. de Geof\'\i sica Aplicada (F.C.A.G) en el marco del
Programa de Incentivos, U.N.L.P. desde el 1/05/98 hasta el 30/04/01.
\ms

\noi 5) Integrante del proyecto: "Propagaci\'on de ondas y flujo, transporte
de contaminantes en el subsuelo terrestre", aprobado y financiado por la
Agencia Nacional de Promoci\'on Cient\'\i fica y Tecnol\'ogica (programa
FONCyT), el que se desarrolla en el Departamento de Geof\'\i sica Aplicada
desde 05/1998 hasta 05/2000.

\ms
\noi 6) Integrante del Proyecto de Investigaci\'on Plurianual (PIP)
``Proyecto para el modelado directo e inverso en  geof\'\i sica aplicada (PROMODIGA)'',
Nro. 0363/98, aprobado y financiado por el Consejo Nacional de Investigaciones Cient\'\i
ficas y Tecnol\'ogicas (CONICET), el que se desarrolla en el Departamento de
Geof\'\i sica Aplicada desde  09/1998 hasta 09/2001.

\ms
\noi 7) Co-directora del proyecto: ``Resoluci\'on num\'erica de problemas
directos e inversos en Geof\'\i sica Aplicada'' (11/G062), desarrollado en el
Depto. de Geof\'\i sica Aplicada (F.C.A.G) en el marco del Programa de
Incentivos, U.N.L.P. desde el 01/01/2001 hasta 31/12/2003.

\ms
\noi 8) Co-directora del proyecto: ``Simulaci\'on num\'erica en Exploraci\'on 
Geof\'\i sica'' (11/G070), desarrollado en el Depto. de Geof\'\i sica Aplicada
(F.C.A.G) en el marco del Programa de Incentivos, U.N.L.P. desde el 01/01/2004
hasta 01/01/2007, prorrogado hasta el 31/12/07. Monto aprox. \$ 20000.
\ms

\noi 9) Integrante del proyecto: ``Modelado Num\'erico para Aplicaciones 
Geof\'\i sicas, 
Agencia Nacional de Promoci\'on Cient\'\i  fica y
Tecnol\'ogica, PICT 2003-13376 (tercera adjudicaci\'on).
Desde el 31/05/05,  (duraci\'on 3 a\~nos).
\ms

\noi 10)Co-directora del proyecto ``Simulaci\'on Num\'erica en
 Geof\'\i sica Aplicada''. Proyecto de Investigaci\'on Plurianual CONICET 
PIP 04 - 5126. Desde octubre de 2005 (duraci\'on 2 a\~nos).
\ms

\noi 11) Integrante del proyecto: ``Metodos num\'ericos en 
Geof\'\i sica Aplicada'', (11/G097), el que se desarrolla en el Depto. de Geof\'\i sica Aplicada
(F.C.A.G) en el marco del Programa de Incentivos, U.N.L.P. desde el 01/01/2008
hasta 31/12/2011. Monto total \$ 40100.
\ms

\noi 12) Integrante del proyecto PIP 2009-2011: ``M\'etodos num\'ericos para la resoluci\'on de problemas en  Geof\'\i sica Aplicada''. Proyecto de Investigaci\'on Plurianual CONICET  
PIP 112-200801-00952. Monto total \$272000.
%luego actualizar las fechas
% PICT enviado en diciembre de 2008 (no fue otorgado)
\ms

\noi 13) Co-directora del proyecto: ``Metodos num\'ericos para la propagaci\'on de ondas y
flujo de fluidos en medios porosos saturados'' (11-G122),  el que se desarrolla en el Depto. de Geof\'\i sica Aplicada
(F.C.A.G) en el marco del Programa de Incentivos, U.N.L.P. desde el 01/01/2012 
hasta 31/12/2015. Director: Dr. Fabio I. Zyserman. Monto estimado: \$ 52600.
% ver codigo asignado  monto a�o 2012 $13146

\ms

\noi 14) Integrante del proyecto ''T�cnicas no-convencionales para el procesamiento de datos s�smicos prestack'',  c�digo 24- 42- 0051,  de Proyectos de Vinculaci�n Tecnol�gica Enrique Mosconi, Convocatoria 2013, (inicio 2014)
Secretaria de Politicas Universitarias, Ministerio de Educaci\'on. Director: Danilo R. Velis.

\ms

\noi 15) Integrante del proyecto PIP 2012-2014: ``Simulaci\'on num\'erica en 
medios porosos saturados''. Proyecto de Investigaci\'on Plurianual CONICET  
PIP 112-201101-00777. Monto total \$135000.
\bigskip


\noindent {\bf 18 - Trabajos  Publicados o Aceptados para Publicar  en
Revistas Peri\'odicas, Actas de Congresos, Libros o Cap\'\i tulos de
Libros.\/} 
%papers

\bigskip
\noindent {\bf En revistas}
\medskip

1.  J.E. Santos, J. Corber\'o, C.L. Ravazzoli, J.Hensley, Reflection and
Transmission Coefficients in Fluid Saturated Porous Media.  Journal of the
Acoustical Society of America, Vol. 91, pags. 1911-1923, (1992) (con
referato).ISSN 0001-4966  Editado como Technical Report \# 151, Center for Applied
Mathematics, Purdue University, Diciembre de 1991.

\sm
2. C.L. Ravazzoli y J.E. Santos, A finite element method for wave propagation
in anelastic media.  Technical Report \#173, Center for Applied Mathematics,
Purdue University, Agosto de 1992 (sin referato).
\sm

{3. C.L. Ravazzoli y J.E. Santos, Consistency analysis for a model for wave
propagation in anelastic media.  Latin American Applied Research, vol.25,
pags. 141-151, (1995) (con referato).} ISSN 0327-0793.
\sm

4. S.E. Oliva y C.L. Ravazzoli, Complex polynomials for the computation of 2-D
gravity anomalies�.  Geophysical Prospecting vol. 45 Nro. 5 (septiembre)
(1997), pags. 809-818 (con referato). ISSN 0016-8025.

\sm
5. M.C. Caama\~no, C.L. Ravazzoli y R. Rodr\'\i  guez, Finite element analysis
of the load effect of the water during a marked swell of the R\'\i  o de La
Plata, Computational Mechanics, New Trends and Applications (CD-ROM), Part
VIII, Section 5 (Geomechanics), Ed. Centro Internacional de M\'etodos Nu
\'ericos en Ingenier\'\i  a, Barcelona, Espa\~na, (1998) (con referato).
Editado como Technical Report \# 98-07, Departamento de Ingenier\'\i  a Mate
\'atica, Fac. de Ciencias F\'\i  sicas y Matem\'aticas, Universidad de
Concepci\'on, Chile, (1998). ISBN 3-540-64605-1.
\sm

{ 6. C.L. Ravazzoli,
Analysis of reflection and transmission coefficients in three-phase sandstone
reservoirs. Journal of Computational Acoustics, vol.9, Nro. 4 (2001), 1437 -
1454 (con referato).} ISSN 0218/396x.
\sm

{7. C.L. Ravazzoli, J.E. Santos, J. M. Carcione, Acoustic and mechanical
response of reservoir rocks under variable saturation and effective pressure.
Journal of the Acoustical Society of America (JASA), Vol.113, pags. 1801-1811,
(2003) (con referato).} DOI: 0.1121/1.1554696. ISSN 0001-4966

\sm 
8. J.E. Santos,  P.M. Gauzellino, C.L. Ravazzoli,
Numerical simulation of waves in poroviscoelastic rocks saturated 
by immiscible fuids. Mec\'anica Computacional, Vol. 21, pags. 652-669,
S.R. Idelsohn, V.E. Sonzogni and A. Cardona (Eds.),
MECOM 2002, Paran\'a, Santa Fe, Octubre de 2002 (con referato).

%Proceedings del 7mo. Congreso Argentino de Mec\'anica Computacional (MECOM 
%2002), 28 al 31 de Octubre de 2002, Paran\'a, Santa Fe.

%\sm
%J.E. Santos, C.L. Ravazzoli, 
%A theory for wave propagation in  porous rocks saturated by 
%two-phase fluids under variable pressure conditions. Enviado a los Proceedings
%of the Royal Society: Mathematical, Physical and Engineering Sciences.

\sm
9. J. E. Santos, C. L. Ravazzoli, P. M. Gauzellino, J. M. Carcione, F.
Cavallini, Simulation of waves in  poro-viscoelastic rocks saturated 
by immiscible fluids. Numerical evidence of a second slow wave.
Journal of Computational Acoustics, Vol. 12\#1, pags. 1-21, (2004) 
(con referato). ISSN 0218/396x.
% JCA vol.12 #1 (march 2004), enviado en octubre 2002,
% aceptacion fonal en feb 2004

\sm
10. J.M. Carcione, J.E. Santos, C.L. Ravazzoli, H.B. Helle,
Wave simulation in partially frozen porous media with fractal freezing
conditions.
Journal of Applied Physics, vol.94, pags. 7839-7847, (2003) (con referato).
DOI: 10.1063/1.1606861. ISSN 0021-8979
%Aceptado en Journal of Applied Physics (Julio/03).

\sm
11. J. M. Carcione, F. Cavallini, J. E. Santos, C. L. Ravazzoli, P. M.
Gauzellino, 
Wave propagation in partially saturated porous media: 
simulation of a second slow wave. Wave Motion, vol. 39, 227-240, (2004)
(con referato). DOI: 10.1016/j.wavemoti.2003.10.001. ISSN 0165-2125.

%enviado a Wave Motion en enero de 2003, luego de rebotar en JCP.
%aceptado en octubre 2003

\sm

{12. J.E. Santos, C.L. Ravazzoli, J.M. Carcione,
A model for wave propagation in a composite solid matrix saturated
by a single phase fluid. Journal of the Acoustical Society of America, 115(6),
2749-2760, (2004) (con referato).} ISSN 0001-4966
% Enviado en junio 2003, revisado y aceptado feb/04.

\sm 

13. C.L. Ravazzoli and J. E. Santos, 
A domain decomposition procedure for the simulation of 
waves  in fluid saturated  composite poroviscoelastic media.
Mec\'anica Computacional, Vol. 23, pags. 652-669,
G. Buscaglia, E. Dari and O. Zamonsky (Eds.), ENIEF 2004, Bariloche, (2004).
8 - 11 de Noviembre de 2004 (con referato).
% enviado 10/08/04

\sm

14. J.M. Carcione, H.B. Helle, J.E. Santos, C.L. Ravazzoli,
A constitutive equation and generalized Gassmann modulus for multi-mineral
porous media.  Geophysics, vol. 70(2), 17 - 26, 2005 (con referato).
ISSN 0016-8033, doi:10.1190/1.1897035
%enviado a Geophysics 10/11/2003, aceptado en  septiembre 04, 
%publicado on-line
% 22/03/05, numero marzo-abril 2005.
\sm

15. J.E. Santos, C.L. Ravazzoli, P.M. Gauzellino, and J.M.Carcione, 
Numerical simulation of ultrasonic waves in reservoir rocks
with patchy saturation and fractal petrophysical properties. 
Computational Geosciences, vol. 9, 1-27, 2005 (con referato) 
ISSN 1420-0597. doi: 10.1007/s10596-005-2848-9
\newline
Publicado en la Technical Report Series
ISC-04-02-MATH, Institute for Scientific Computation, Texas A\&M University,
marzo 2004. (www.isc.tamu.edu/iscpubs/iscreports.html).
% enviado a CG el 10/03/04 luego de ser rechazado por Wave Motion donde fue
% enviado en febrero 2004. Aceptado en CG 28/02/05

\sm

16. C. L. Ravazzoli and J.E. Santos, 
A theory for wave propagation in  porous rocks saturated by 

two-phase fluids under variable pressure conditions.   
Bolletino di Geofisica Teorica ed Applicata, vol. 46/4, 261-285,(2005).
ISSN 0006-6729
Publicado en la Technical Report Series
ISC-04-05-MATH, Institute for Scientific Computation, Texas A\&M University,
abril 2004. (www.isc.tamu.edu/iscpubs/iscreports.html).
% enviado a BGTA (mayo 2004), aceptado feb 05, saldra 
% en el Vol.46/4 December 2005

\sm

17. J. E. Santos, C. L. Ravazzoli and  J. Geiser,
On the static and dynamic behavior of fluid saturated composite 
porous solids; a homogenization approach.
International Journal of Solids and Structures , vol 43, 1224-1238, 2006.
doi:10.1016/j.ijsolstr.2005.04.018. ISSN 0020-7683.
Publicado en la Technical Report Series
ISC-04-11-MATH, Institute for Scientific Computation, Texas A\&M University,
Octubre 2004.
%enviado en nov. 2004, aceptado online 13/06/05
\sm

18. D. S. Cersosimo, C. L. Ravazzoli, R. Garc\'\i a Mart\'\i nez,  Inversi\'on 
s\'\i smica de un modelo te\'orico calculado sobre un horizonte s\'\i smico
utilizando redes neuronales. Bolet\'\i n de Informaciones Petroleras A.A.G.G.P.,
 vol. 1, 44 - 51, agosto de 2005.
\sm

19. C.L. Ravazzoli, J.G. Rubino and J. E. Santos, 
Numerical analysis of wavefields in composite frozen porous media.
Mec\'anica Computacional, Vol. 24, pags. 2413-2427 (Ed. Axel Larrateguy),
MECOM 2005, Buenos Aires, (2005)(con referato). ISSN 1666-6070
\sm

20. N. Maltagliatti y C.L. Ravazzoli,  
An\'alisis comparativo de m\'etodos de estimaci\'on espectral 
para la determinaci\'on de profundidad al basamento
magn\'etico en exploraci\'on geof\'\i sica
Mec\'anica Computacional, Vol. 24, pags. 3131-3147,
(Ed. Axel Larrateguy), MECOM 2005, Buenos Aires, (2005)(con referato).
ISSN 1666-6070.
\sm

21. D. S. Cersosimo, C. L. Ravazzoli, R. Garc\'\i a Mart\'\i nez,
Identification of velocity variations in a seismic cube using neural networks.

%IFIP International Federation for Information Processing, Volume 218, 
Professional Practice in Artificial Intelligence, IFIP 2006 19th. World 
Computer Congress, J. Debenham (Ed.), Springer, Boston, pp. 11-19. 
ISSN 1571-5736.

\sm

22. J. Germ\'an Rubino, Claudia Ravazzoli, Juan Santos, 
Reflection and transmission of waves in composite porous media: a
quantification of slow waves conversions. Journal of the Acoustical Society 
of America, 120(5), 2425-2436, (2006). ISSN 0001-4966.
%enviado a JASA 16/12/2005. En prensa 23/08/06. Publicado Noviembre 2006
\sm

23. J. Germ\'an Rubino, Claudia Ravazzoli, Juan Santos, 
A numerical procedure to estimate the effective moduli in highly heterogeneous 
fluid-saturated porous media. XVI Congreso sobre M\'etodos Num\'ericos y sus
 Aplicaciones ENIEF 2007-MACI 2007. Mec\'anica Computacional vol. 26, 1747-1773, 
2007. ISSN 1666-6070
%enviado el 06/07/2007
\sm

24. J. Germ\'an Rubino, Claudia Ravazzoli, Juan Santos, 
%The role of Biot slow waves in gas hydrate bearing sediments. Enviado a
%Journal of Geophysical Research .
%enviado a JGR 23/11/2006.
Biot-type scattering effects in gas hydrate-bearing sediments. 
Journal of Geophysical Research (Solid Earth), 113, B06102 (16 pags.), 2008.
%version revisada a JGR 11/07/2007. Aceptacion final 15/01/08, publicado
% 5 de junio 2008

\sm
25. J. Germ\'an Rubino, Claudia L. Ravazzoli and Juan E. Santos, Equivalent viscoelastic
solids for heterogeneous fluid-saturated porous rocks. 
Geophysics, 74(1), N1-N13, (2009). ISSN 0016-8033.
% 08/04/08, en realidad es revision de la primer version enviada el 21/09/07.
% aceptacion en julio 2008, con cambios menores. Aceptacion final: 03-09-08

\sm
26. Juan E. Santos, J. Germ\'an Rubino and Claudia L. Ravazzoli, 
A Numerical Upscaling
Procedure to Estimate Effective Plane Wave  and Shear Moduli  in
Heterogeneous Fluid-saturated Poroelastic Media. Publicado en 
Computer Methods in Applied Mechanics and Engineering (CMAME), 198, 2067-2077, (2009). 
ISSN: 0045-7825
% disponible online 20/03/09
% fecha envio: 15/01/08.

\sm
27. Claudia L. Ravazzoli and Juli\'an G\'omez, AVA Seismic reflectivity analysis in carbon dioxide accumulations: sensitivity to CO$_2$ phase and saturation.
Journal of Applied Geophysics, 73, 93-100, 2011. DOI 10.1016/j.jappgeo.2010.11.010

%Enviado a Journal of Applied Geophysics,  el 28/09/09.
% aceptado 24 de noviembre de 2010. Online 2 dic 2010,
% publicado 10/02/2011
\sm
28. Juli\'an L. G\'omez and Claudia L. Ravazzoli, Reflection Characteristics of 
Carbon Dioxide Transition Layers. Geophysics, 77(3), D75-D83, (2012).

%Enviado a Geophysics, 23/10/2011, aceptado ~01/02/2012

\sm
29. Juli\'an L. G\'omez and Claudia L. Ravazzoli, Seismic reflectivity of a 
carbon dioxide flux. Mec\'anica Computacional,31, 631-649, 2012.
%proceedings congreso MECOM 2012.
%Enviado en Julio 2012, aceptado ~17/09/2012

\sm
30. Juli\'an L. G\'omez and Claudia L. Ravazzoli, Seismic spectral monitoring of  
carbon dioxide in a geological reservoir. Mec\'anica Computacional,32, 1183 - 1197, 2013.
%proceedings congreso ENIEF 2013.
%Enviado en Julio 2013, 

\sm
31.Cers\'osimo, S., Ravazzoli, C., Garc\'\i a Mart\'\i nez, R. 
Prediction of lateral variations in reservoir properties
throughout an interpreted seismic horizon using ANN. The Leading Edge, 35(3), 
265-269, 2016.
% First Published on March 02, 2016 
%enviado a fines de Diciembre de 2014. Aceptado el 11/08/15. En prensa para el volumen de Febrero 2016.

\bs
 
\noi{\bf En actas de congresos}

\bs
1.  C.L. Ravazzoli, J. Douglas Jr., J.E. Santos, D. Sheen, On the solution of
the equations of motion for nearly elastic solids in the frequency domain. 
Actas de la IV Reuni\'on de Trabajo en Procesamiento de la Informaci\'on y
Control C.N.E.A. del 18 al 22 de Noviembre de 1991, pags. 231-235 (con
referato).  Editado como Technical Report \# 164, Center for Applied
Mathematics, Purdue University, Septiembre de 1991.

\sm

2. C.L. Ravazzoli, Propagaci\'on de Ondas en Rocas Saturadas. IV Congreso
Internacional de Ciencias de la Tierra, Santiago, Chile, del 5 al 9 de Agosto
de 1996.\newline
\noi Aclaraci\'on: si bien este trabajo fue aceptado, debido a razones
presupuestarias no se publicaron los trabajos completos.

\sm

3.  M.C. Caama\~no, C.L. Ravazzoli y R. Rodr\'\i  guez, A numerical model for
the study of the load effect of the water in the area of the River Plate.
Geodesy on the Move, Proceedings of the International Association of Geodesy
Scientific Assembly, Rio de Janeiro, Brasil, 3 al 9 de Septiembre de 1997,
pags. 411 - 416 (con referato).

\sm

4. L. Guarracino, C.L. Ravazzoli y J.E. Santos, Modelado de Transporte de
contaminantes en las zonas saturada y no saturada utilizando m\'etodos de
elementos finitos. Actas del 4to. Congreso Latinoamericano de Hidrolog\'\i  a
Subterr�nea, Montevideo, Uruguay, 16 al 20 de Noviembre de 1998, vol. 1, pags.
156 - 168 (con referato).

\sm

5. C.L. Ravazzoli y J.E. Santos, Compressibility analysis of Berea sandstone 
versus saturation and effective pressure, 
{62nd. European Association of Geoscientists and Engineers Conference and
technical Exhibition} Extended Abstracts, volume 1 (paper D-36),
Glasgow (Escocia), 29 de Mayo -- 2 de Junio de 2000.

\sm
6. J.M. Pi Alperin y C.L. Ravazzoli, Modelado y ajuste de velocidades s\'\i
smicas en ambientes sedimentarios de reservorio", Actas (en CD-ROM) de la
20ma. Reuni\'on  Cient\'\i  fica  de  la  Asociaci\'on
Argentina de Geof\'\i sicos y Geodestas , Mendoza, 25 al 29 de septiembre de
2000, pags. 312 -- 314.

\sm
7. C.L. Ravazzoli, J. M. Carcione, J.E. Santos, H.B. Helle, Acoustic
properties of an overpressured sandstone saturated by immiscible fluids,
Society
of Exploration Geophysicists, International Exposition and 71st. Annual
Meeting, San Antonio, Texas, 9 al 14 de septiembre de 2001. Expanded Abstracts
(CD-ROM) (4 pags.) (con referato). ISSN 1052-3812

\sm
8. J.E. Santos, C.L. Ravazzoli, J.M. Carcione, P.M. Gauzellino, F. Cavallini,
Prediction and simulation of a second slow wave in partially saturated porous
media, {European Association of Geoscientists and Engineers 64th. Conference
and  Exhibition, Florencia, Italia, 27 -30 Mayo de 2002.
Abstracts Book and CD-ROM} (4 pags.) (con referato).
\sm

9. D. S. Cers\'osimo, C. L. Ravazzoli, R. Garc\'\i a Mart\'\i nez,  Inversi\'on 
s\'\i smica de un modelo te\'orico calculado sobre un horizonte s\'\i smico
utilizando redes neuronales. 3er. Convenci\'on T\'ecnica de la ACGGP, ``La
inversi\'on en el conocimiento geol\'ogico'', Bogot\'a, 29/09 al 02/10/04.

\sm

10. J. G. Rubino, C. L. Ravazzoli and J. E. Santos, 
Modeling  and inversion of sonic P and S wave velocities at the 
Mallik 5L-38 Gas Hydrate Research Well. Expanded Abstract A029 (4 pags.) 
EAGE/EAGO/SEG International Conference and Exhibition, Saint Petersburg 2006,
 Russia, 16 - 19 de Octubre de 2006. (con referato). ISBN 90-73781-64-7.

\sm

11. J. E. Santos,  J. G. Rubino and C. L. Ravazzoli,
Modeling  the reflection coefficients and slow wave 
mode conversions at the top and bottom of a gas-hydrate bearing interval.
Expanded Abstract 25 (4 pags.) Society of Exploration Geophysicists, International 
Exposition and 76th. Annual
Meeting, New Orleans,  1 al 6 de Octubre de 2006. (con referato). ISSN 1052-3812

\sm
12. J. E. Santos, C. L. Ravazzoli and J. G. Rubino, 
Statistical analysis of the effective velocity and mesoscopic attenuation in 
patchy saturated  porous media.
Expanded Abstract (5 pags.) Society of Exploration Geophysicists, International 

Exposition and 77th. Annual Meeting, San Antonio, Texas, 23 al 28 de Septiembre 
de 2007, 26, pags. 2708-2712 (con referato). ISSN 1052-3812

\sm
13. L. Mariana Longo and Claudia L. Ravazzoli, Souce Location estimation from
noisy magnetic data using Euler's homogeneity equation. American Geophysical Union 
AGU 2007 Joint Assembly, Acapulco, M\'exico, 22 al 25 de Mayo de 2007.
Presentado como poster en la sesi\'on: ``GP06 - New discoveries in magnetic and gravity anomaly 
interpretation  methodologies and their innovative application for geologic, environmental,
exploration and planetary scale potential field data''. Resumen publicado en
EOS Trans. AGU, 88(23), Jt. Assem. Suppl., Abstract GP31A-03. ISSN 0096-3341.

\sm
14. J. E. Santos,  J. G. Rubino and C. L. Ravazzoli
Modelling mesoscopic attenuation in a highly heterogeneous Biot's medium 
employing an effective viscoelastic model.
Expanded Abstracts, 27(1), 2112-2116, Society
of Exploration Geophysicists, International Exposition and 78th. Annual
Meeting, Las Vegas, 2008. (con referato). ISSN 1052-3812
%enviado 08/04/08

\sm


15. L. Mariana Longo, Claudia L. Ravazzoli, Massimo Chiappini, Proyecto de interpretaci\'on 
de datos aerogravim\'etricos y magn\'eticos en el Volc\'an Auca Mahuida.
4to. Encuentro Internacional E-ICES 4, International Center for Earth Sciences,
Malarg\"ue, Mendoza,29 al 31 de Octubre 2008. Actas E-ICES 4,Publicado en CD-ROM.
Comisi\'on Nacional de Energ\'\i a At\'omica, Ediciones T\'ecnicas, 2009.
ISBN 978-987-1323-11-1.

\sm

16. D.S. Cers\'osimo, C.L. Ravazzoli, R. Garc\'\i a Mart\'\i nez, Predicci\'on 
de cuerpos arenosos con redes neuronales. ``Modelado Geol\'ogico'', 
pags. 263-267, VII Congreso de 
Exploraci\'on y Desarrollo de Hidrocarburos 2008, Instituto Argentino del 
Petr\'oleo y del Gas. ISBN 978-987-9139-49-3.

17. J.L. G\'omez,  C.L. Ravazzoli, Theoretical AVA analysis for geological CO$_2$ 
sequestration, Actas de la 24ta. Reuni\'on  Cient\'\i  fica  de  la  Asociaci\'on
Argentina de Geof\'\i sicos y Geodestas , Mendoza, 14 al 17 de abril de
2009, 5 pags, \newline http://www.aagg2009.org/espanol/contents/trabajos. 
Libro de Res\'umenes pags. 154-155.
\sm

18. J. G. Rubino and C. L. Ravazzoli, Seismic attenuation and velocity dispersion
in rocks containing heterogeneous distributions of carbon dioxide, 24ta. Reuni\'on  Cient\'\i  fica  de  la  Asociaci\'on
Argentina de Geof\'\i sicos y Geodestas , Mendoza, 14 al 17 de abril de
2009. Libro de Res\'umenes pags. 156-157.
\sm

19. C.L. Ravazzoli and J.L. G\'omez, Seismic reflectivity analysis in CO$_2$ 
sequestration problems, {Segundo Workshop sobre Inversi\'on y Procesamiento de Se\~nales
en Exploraci\'on S\'\i smica IPSES 09, F.C.A.G.}, 14 de Agosto de 2009.
\sm

20. S. Picotti, J.E. Santos, J.M. Carcione, D. Gei and C.L. Ravazzoli, A physics and numerical methodology for modeling CO$_2$ geological storage. 
SEG 2009 Summer Research Workshop on CO$_2$  Sequestration Geophysics, 23-27 August 2009 Banff, Canad\'a. Proceedings (11 pags).
%OJO, cuando escribi a la SEG me dijeron que no publican formalmente!!
\sm

21. C.L. Ravazzoli, J.L. G\'omez, F. Carozzi, Seismic reflectivity behavior of CO2 bearing layers, AGU Meeting of the Americas 2010,
 EOS Transactions AGU 91(26), AGU 2010 Joint Assembly Supplements abstract 853568 (S21A-06). ISSN 0096-3941

\sm

22. Juli\'an L. G\'omez, Claudia L. Ravazzoli and Fernanda E. Carozzi,
Generalized reflectivity of CO$_2$ partially saturated layers.
 Society
of Exploration Geophysicists, International Exposition and  Annual
Meeting, Denver 2010, Expanded Abstract 29(1), 488-492. ISSN 1052-3812.
%enviado 07/04/20010 
% Denver 17-22 Octubre 2010
\sm

23. Stefano Picotti, Juan  E. Santos, 
Jos\'e M. Carcione, Davide Gei and Claudia L. Ravazzoli, A  finite element method to model attenuation and dispersion effects in highly heterogeneous fluid-saturated porous media,  Theoretical and Computational Acoustics 2009, Steffen Marburg Ed., ICTCA 09 Conference Proceedings (Dresden, Germany 7-11 Sept 2009), 235-246. Publicado Sept 2010 (aprox.)
% enviado en Nov 2009
% publicado aprox sept 2010. A diferencia del del ICTCA99 que lo puse como
% contribucion a capitulo de libro, esto fue editado por una universidad 
%alemana desconocida, no una editorial. No tiene ISBN. Por esto no lo pongo como
%"libro"


\sm

24. Juli\'an L. G\'omez, Claudia L. Ravazzoli, Detectability analysis of miscible CO2 in a CCS/EOR oil reservoir, VIII Congreso de 
Exploraci\'on y Desarrollo de Hidrocarburos 2011, Trabajos T\'ecnicos, 489-503, 
Instituto Argentino del Petr\'oleo y del Gas. ISBN 978-987-9139-61-5. 
%Enviado en Junio de 2011, aceptado 04/10/2011

\sm

25. D.S. Cers\'osimo, C.L. Ravazzoli, R. Garc\'\i a Mart\'\i nez, Predicci\'on 
de geometr\'\i as de meandros, cuerpos de dunas y espesores a lo largo de un
volumen s\'\i smico generado a partir de una superficie previamente interpretada.
 VIII Congreso de Exploraci\'on y Desarrollo de Hidrocarburos 2011,"Simposio de Geof\'\i sica:
 Integraci\'on, acercando la ond\'\i cula al tr\'epano", 55-69, Instituto Argentino del Petr\'oleo y del Gas. ISBN 978-987-9139-60-8. 

\sm

26. Cers\'osimo, S., Ravazzoli, C., Garc\'\i a Mart\'\i nez, R. 2011. Predicci\'on de las geometr\'\i as 
de ambientes sedimentarios deltaicos y estuarinos por medio de modelado sint\'etico de velocidades. 
 Actas del XVIII Congreso Geol\'ogico Argentino. 1106-1107. ISBN 978-987-22403-4-9. 
\sm

27. Cers\'osimo, S., Ravazzoli, C., Garc\'\i a Mart\'\i nez, R. 2011. Non-classical use of neural networks to predict petrophysical properties in a seismic cube.  Proceedings of II International Congress on Computer Science and Informatics (INFONOR-CHILE 2011). Pp. 104-110. ISBN 978-956-7701-03-

\sm

28. Juli\'an L. G\'omez, Claudia L. Ravazzoli, Modelling the reflectivity of a carbon dioxide transition zone, presentado como poster en {\it 3rd. EAGE CO2 Geological Storage Workshop:
Understanding the Behaviour of CO2 in Geological Storage Reservoirs}, 26 - 27 March 2012, Edimburgh, Scotland. 
Publicado en Workshop Proceedings, pag.114-118 (www.earthdoc.org).

% enviado 01/10/2011
%aceptado 07/11/2011
29. Juli\'an L. G\'omez, Claudia L. Ravazzoli, Monitoring of CO2 in a geological reservoir with the seismic peak frequency attribute, Expanded Abstracts, Society
of Exploration Geophysicists, International Exposition and  Annual
Meeting, Denver 2014, 4965-4970.

\sm
30. Oksana Bokhonok, Claudia L. Ravazzoli, Sensitivity study and comparative analysis of elastic properties and anisotropy coefficients in organic shales.
10mo. Encuentro Internacional E-ICES 10, International Center for Earth Sciences, Buenos Aires, 3 al 6 de Noviembre de 2014. Actas de Trabajos Completos,
51-63.  ICES Nodo Argentina ISBN: 978-987-1323-39-5, 2014. 
http://www.cnea.gov.ar/cac/ices.

\sm

31. Liliana Guevara, Claudia L. Ravazzoli,   Aplicaci\'on de la Transformada-S para  el
an\'alisis tiempo-frecuencia de sismogramas ac\'usticos en ambientes de reservorio (p\'oster), VIII Workshop Wavelets y Teor\'\i a de la Informaci\'on.
Facultad de Ciencias Exactas e Ingenier\'\i a UNLP, La Plata 9 al 11 de agosto de 2016. Libro de Res\'umenes, pag.36-37. ISBN: 978-950-34-1363-0.http://sedici.unlp.edu.ar/handle/10915/53900

\sm
32. Oksana Bokhonok and Claudia L. Ravazzoli. The influence of kerogen content on some brittleness indicators in a rich organic shale: comparative analysis of different approaches (p\'oster). 
Actas del VII Simposio Brasileiro de Geof\'\i sica, SBGf - Sociedade Brasileira de Geof\'\i sica, Ouro Preto, 25 al 27 de octubre de 2016 (4 pags.). ISSN 2179-0965
%29/04/16

\sm
33. Claudia L. Ravazzoli, Gonzalo Blanco y Juan C. Soldo. An\'alisis petro-el\'astico de velocidades P y S en muestras de la formaci\'on Vaca Muerta. XXVIII Reuni\'on  Cient\'\i  fica  de  la  Asociaci\'on Argentina de Geof\'\i sicos y Geodestas, La Plata, 17 al 21 de Abril  de 2017, Libro de Res\'umenes Expandidos p. 77-80, http://sedici.unlp.edu.ar/handle/10915/60718 y Libro de Res\'umenes, pag.84. \\ 
http://sedici.unlp.edu.ar/handle/10915/60712, ISBN 978-950-34-1471-2
\sm

34. Liliana Guevara y Claudia L. Ravazzoli,  An\'alisis tiempo-frecuencia y comportamiento de los picos espectrales en sismogramas de reservorios finos (p\'oster). XXVIII Reuni\'on  Cient\'\i  fica  de  la  Asociaci\'on Argentina de Geof\'\i sicos y Geodestas, La Plata, 17 al 21 de Abril  de 2017(4 pags.). Libro de Res\'umenes, pag.62, http://sedici.unlp.edu.ar/handle/10915/60712.  ISBN: 978-950-34-1470-5.\\
Trabajo distinguido con el 
\emph{Premio al mejor p\'oster de Geof\'\i sica Aplicada} del congreso.

\bs
\noindent{\bf Trabajos enviados}

\medskip

Oksana Bokhonok and Claudia L. Ravazzoli. Rock-physics-based predictions of brittleness indicators in a rich organic shale. Aceptado para el 
First EAGE/IAPG Workshop on Geophysics for Unconventionals
Buenos Aires, Argentina, 3-6 November 2015 (suspendido por razones presupuestarias).
%Juli\'an L. G\'omez, Claudia L. Ravazzoli, A Peak frequency study for seismic 
%CO$_2$ monitoring. Enviado a Geophysics el 22/04/2013.

\medskip
Gonzalo Blanco, Claudia L. Ravazzoli, Juan Carlos Soldo, Modelos de velocidad para shales org\'anicas de Vaca Muerta calibrados con datos de laboratorio y de pozos (trabajo completo). Enviado al
X Congreso de Exploraci\'on y Desarrollo de Hidrocarburos IAPG - CONEXPLO 2018.   
\medskip


Guido Panizza, Claudia L. Ravazzoli, Juan Carlos Soldo, Aproximaci\'on de velocidades  anis\'otropas en shales mediante modelos de f\'\i sica de rocas heur\'\i sticos (res\'umen). 
Enviado al
X Congreso de Exploraci\'on y Desarrollo de Hidrocarburos IAPG - CONEXPLO 2018.  

%J. E. Santos, J. Geiser and C. L. Ravazzoli,  
%On the static and dynamic behavior of fluid saturated composite 
%porous solids; a homogenization approach.ISSN 0020-7683.
%Enviado a International Journal of Solids and Structures (noviembre 2004).
%Publicado en la Technical Report Series
%ISC-04-11-MATH, Institute for Scientific Computation, Texas A\&M University,
%Octubre 2004.
% reboto en Int. J. Engineering Science (24 septiembre 2004). 
%\medskip

\begin{comment}
Cers\'osimo, S., Ravazzoli, C., Garc\'\i a Mart\'\i nez, R. 
Prediction of lateral variations in reservoir properties
throughout an interpreted seismic horizon using ANN. Enviado a The Leading
Edge (SEG), a fines de Diciembre de 2014. Aceptado el 11/08/15. En prensa para el volumen de Febrero 2016.

Oksana Bokhonok and Claudia L. Ravazzoli. Rock-physics-based predictions of brittleness indicators in a rich organic shale. Aceptado para el 
First EAGE/IAPG Workshop on Geophysics for Unconventionals
Buenos Aires, Argentina, 3-6 November 2015 (suspendido y reprogramado para 2016).
\end{comment}

 
\bigskip

\noindent{\bf En preparaci\'on}

\medskip

Claudia L. Ravazzoli, Gonzalo Blanco, 
Exploring correlations between thermal maturity indicators and kerogen physical properties 
in Vaca Muerta organic shales using rock physics: preliminary results. Para ser enviado a 
\emph{Special Issue for Advances in Geo-Energy Research (AGER): 
Shale Thermal Maturity Issues and Modelling}.

\medskip
\noi {\bf Cap\'\i tulos de Libros}

\medskip

C.L. Ravazzoli, Analysis of reflection and transmission coefficients in 
three-phase sandstone reservoirs. Journal of Computational Acoustics vol. 9(4) 2001,
1437-1454, incluido en el capitulo "Wave propagation Theory" del libro 
"Theoretical and
Computational Acoustics '99" (CD-ROM). Eds. G\'eza Seriani, Ding Lee, 
World Sci. Publ. Singapur (2004).
ISBN 981-238-447-2.

\medskip

C.L. Ravazzoli and J.L. G\'omez, Seismic Reflectivity in Carbon Dioxide Accumulations: a Review. Publicado en el libro "Carbon Sequestration and valorization", Chapter 12, pags. 343-364. Eds. C. Morgado, V. Esteves, Intech Open  (2014).  ISBN 978-953-51-1225-9. {http://dx.doi.org/10.5772/57034} \newline
{\it http://www.intechopen.com/articles/show/title/seismic-reflectivity-in-carbon-dioxide-\newline accumulations-a-review }
% enviado 09/08/2013, aceptado 20/09/2013, on-line 12/03/14


\bigskip
\noindent {\bf 19 - Trabajos de transferencia/extensi\'on efectuados:\/}
\medskip

 Participante del proyecto de investigaci\'on y transferencia: Detection of
Overpressured Zones from Seismic and Well Data, en el marco del Proyecto
Joule,  Union Europea, en car\'acter de Asistencia Externa al Osservatorio
Geofisico Sperimentale, Trieste (Italia), desde el 01/12/97 al 01/12/00.
Reportes t\'ecnicos anuales aprobados por la Uni\'on Europea en Diciembre de
1998 y Diciembre de 1999.

\medskip

Evaluaci\'on T\'ecnica de un proyecto  CAEFIPP 2005, Programa de 
Modernizaci\'on Tecnol\'ogica II, Agencia Nacional de Promoci\'on Cient\'\i fica y 
Tecnol\'ogica, Octubre de 2005.

\ms

Integrante grupo responsable del proyecto: ``CO$_2$ Geological Storage: research into monitoring 
and verification technology CO2REMOVE'', financiado en el marco del 6th. Framework 
Programme Sustainable Energy Systems, European Community and The Netherlands 
Organization (TNO),
contrato 518350(SES6), desde el 01/09/2007 al 31/03/2011. Monto total 90100 Euros.

\ms
Evaluaci\'on de un proyecto PICT para la Agencia Nacional de Promoci\'on Cient\'\i fica y Tecnol\'ogica, en Octubre de 2013.

\ms
Evaluaci\'on de un proyecto PICT para la Agencia Nacional de Promoci\'on Cient\'\i fica y Tecnol\'ogica, en Diciembre de 2016.

\bigskip
\noindent {\bf 20 - Traducciones:\/}
\bigskip

\noindent {\bf 21 - Formaci\'on y Direcci\'on de Recursos Humanos:\/}
\bigskip

\noindent {\bf 21.1 Becarios.\/}
\medskip

Co-directora de la Beca de Perfeccionamiento y Prorroga otorgada por el
 CONICET a la Geof. 
Maria Cristina Caama\~no en el tema: {\it Aplicaciones de los datos argentinos de 
mareas terrestres}.
Director: Ing. Juan Carlos Usandivaras (F.C.A.G.). Per\'\i odo: abril de 1996 - abril 1999.
Lugar de trabajo: Facultad de Ciencias Astron�micas y Geof. U.N.L.P.

\medskip

Co-Directora de la Beca Interna Doctoral (Beca Interna de postgrado tipo I)
otorgada por el CONICET al Geof. Rubino Jorge Germ\'an en el tema: {\it 
An\'alisis y aplicaci\'on de modelos f\'\i sicos para la 
propagaci\'on de ondas ac\'usticas en medios porosos multif\'asicos}.
Director: Dr. Juan E. Santos. Fecha de inicio: 01/04/2005. Lugar de trabajo:
 Facultad de Ciencias Astron\'omicas y Geof. U.N.L.P.
\ms

Directora de la Beca Postdoctoral Interna otorgada por el CONICET al Dr. 
Rubino Jorge Germ\'an, en el tema {\it Propagaci\'on de ondas s\'\i smicas 
en medios porosos heterog\'eneos}. Periodo: 01/04/2009 -30/04/2010. Lugar de trabajo:
 Facultad de Ciencias Astron\'omicas y Geof. U.N.L.P.
\ms

Directora de la Beca Interna Doctoral (Beca Interna de postgrado tipo I)
otorgada por el CONICET al Geof. Juli\'an Luis G\'omez en el tema: {\it 
Desarrollo y aplicaci\'on de modelos f\'\i sicos y num\'ericos para la 
propagaci\'on de ondas s\'\i smicas en medios multif\'asicos} 
Periodo: 01/04/2009-31/03/2012. Lugar de trabajo:
Facultad de Ciencias Astron\'omicas y Geof. U.N.L.P.
\ms

Directora de la Beca Interna Doctoral (Beca Interna de postgrado tipo II)
otorgada por el CONICET al Geof. Juli\'an Luis G\'omez en el tema: {\it  
Modelado y an\'alisis de indicadores para el monitoreo s\'\i smico de 
di\'oxido de carbono en el subsuelo.}
Periodo: 01/04/2012-31/03/2014. Lugar de trabajo:
Facultad de Ciencias Astron\'omicas y Geof. U.N.L.P.
\ms

Co-Directora de la Beca Postdoctoral Interna otorgada por el CONICET al Dr. 
Juli\'an Luis G\'omez en el tema: {\it M\'etodos s\'\i smicos espectrales
para la caracterizaci\'on de reservorios convencionales y no convencionales}
(Director Dr. Danilo Velis).  
Periodo: 01/04/2014 -31/03/2016, prorrogada. Lugar de trabajo:
 Facultad de Ciencias Astron\'omicas y Geof. U.N.L.P.

\ms

Directora de la Beca Interna Doctoral co-financiada otorgada por CONICET-YTec,
 al Geof. Guido Panizza en el tema: {\it  An\'alisis y aplicaci\'on de 
modelos de f\'\i sica de rocas multiescala en rocas reservorio no convencionales 
argentinos}. Co-director: Dr. Juan C. Soldo.
Per\'\i odo: 01/07/2017-01/07/2019 (renovable por otros dos a\~nos). Lugar de trabajo:
YPF Tecnolog\'\i a.
\bigskip

\noindent{\bf 21.2 Direcci\'on de tesis: terminadas y en curso:\/}
\medskip

\noi{\bf Tesis de Grado}
\sm

1. Direcci\'on de la Tesis de Grado del alumno de la carrera de Geof\'\i sica Juan
M. Pi Alperin, en el tema: {\it F\'\i sica de rocas aplicada al modelado e
interpretaci\'on de velocidades en ambientes sedimentarios}, F.C.A.G.
Aprobada el 20/12/99.
\medskip

2. Co-Direcci\'on de la Tesis de Grado de la alumno de la carrera de Geof\'\i sica
Laura Mariana Longo, en el tema: {\it Prospecci\'on de Hidrocarburos en la
Cuenca Neuquina aplicando m\'etodos s\'\i smicos de reflexi\'on}, (Director:
Ing. Alberto Chernikoff), F.C.A.G. Aprobada el 22/09/2000.
\medskip

3. Direcci\'on de la Tesis de Grado de la alumna de la carrera de 
Geof\'\i sica 
Natalia Maltagliatti, en el tema: {\it Estimaci\'on  de la profundidad al basamento
magn\'etico mediante m\'etodos espectrales}.
Plan de Tesis aprobado por resoluci\'on del Consejo Acad\'emico 
de la F.C.A.G. del 11/11/2004. Aprobada el 15/11/2006.
\medskip

4. Direcci\'on de la Tesis de Grado en Geof\'\i sica de la alumna Fernanda E. Carozzi, en el tema: {\it Modelado y an\'alisis de la reflectividad s\'\i smica generalizada en medios estratificados}.  Aprobada el 03/12/2010.

\medskip
5. Direcci\'on de la Tesis de Grado en Geof\'\i sica de la alumna Oksana Bokhonok,
en el tema: {\it Modelado comparativo de coeficientes poroel\'asticos 
anis\'otropos efectivos y magnitudes relacionadas para la caracterizaci\'on de lutitas org\'anicas}.  Aprobada el 12/06/2015.

\medskip
6. Direcci\'on de la Tesis de Grado en Geof\'\i sica de la alumna Liliana Guevara,
en el tema: {\it Comportamiento espectral y variaciones de frecuencia pico en sismogramas de reflexi\'on poroel\'asticos}. 
Aprobada el 30/03/2016.

\medskip
7. Direcci\'on de la Tesis de Grado en Geof\'\i sica del  alumno Gonzalo Blanco,
en el tema: {\it Modelos petro-el\'asticos de velocidades en la formaci�n Vaca Muerta}, 
trabajo en curso desde octubre de 2016.

\bs

\noi{\bf Tesis Doctorales}
\sm

1. Co-Direcci\'on de la Tesis Doctoral en Geof\'\i sica del Geof. Dar\'\i o Sergio
Cers\'osimo, en el tema: {\it Inversi\'on de datos s\'\i smicos con m\'etodos no
convencionales}. Director: Dr. Ram\'on Garc\'\i a Mart\'\i nez.
Plan de Tesis aprobado en Expte. 1100-5171/02 por resoluci\'on del Consejo
 Acad\'emico de la F.C.A.G. del 26/02/04. Re-inscripci\'on al doctorado en el 
tema: \emph{Predicci\'on  de informaci\'on geof\'\i sica basada en datos s\'\i smicos interpretados usando redes neuronales}. Expte 1100-1624/16, Res.CD 21/04/2016.

\medskip
%Co-Direcci\'on de la Tesis Doctoral en Geof\'\i sica de la Geof. Laura
% Mariana Longo
%(en curso) (Directora: Dra. Marta Ghidella, Instituto Antartico Argentino).
%\medskip


2. Co-Direcci\'on de la Tesis Doctoral en Geof\'\i sica del Geof. Jorge Germ\'an 
Rubino, en el tema: {\it Atenuaci\'on y dispersi\'on de ondas s\'\i smicas 
en medios porosos saturados altamente heterog\'eneos}, 
%{\it Desarrollo  y aplicaci\'on de modelos f\'\i sicos para la 
%propagaci\'on de ondas ac\'usticas en medios porosos multif\'asicos},
dirigida por Dr. Juan E. Santos.
Plan de Tesis aprobado en Expte. 1100-724/05 (Act. 533/05) por 
resoluci\'on del Consejo Acad\'emico de la F.C.A.G. del 24/11/05.
Fecha de defensa y aprobaci\'on: 09/12/2008.
{\sl 
Premio a la mejor Tesis Doctoral en Geof\'\i sica Pura o Aplicada, realizada en Universidades Espa\~nolas o de Iberoam\'erica, en el marco de la XV Convocatoria de Premios de Geof\'\i sica de la Fundaci\'on Espa\~nola J. Garc\'\i a-Si\~neriz, correspondiente al curso 2008-2009.}
\ms

3. Directora de la Tesis Doctoral en Geof\'\i sica del Geof. Juli\'an Luis G\'omez en el tema: {\it An\'alisis de la reflectividad s\'\i smica e indicadores asociados
para el monitoreo 
de reservorios de di\'oxido de carbono.}
Inscripci\'on al doctorado aprobada por Expte. 1100-00362/09 por resoluci\'on del Consejo
 Acad\'emico de la F.C.A.G. del 17/12/2009. Fecha de defensa y aprobaci\'on: 21/03/2014. Calificaci\'on : 10.


4. Directora de la Tesis Doctoral en Geof\'\i sica del Geof. Guido Panizza en el tema: {\it  Implementaci\'on, an\'alisis y aplicaci\'on de 
modelos de f\'\i sica de rocas para reservorio no convencionales}. Co-director: Dr. Juan C. Soldo. Inscripci\'on al doctorado aprobada por Expte. 1100-3234/17 por resoluci\'on del Consejo
 Acad\'emico de la F.C.A.G. del 28/03/2018.

\bs

\noindent {\bf 21.3 Direcci\'on de docentes--investigadores:\/}

\medskip

Geof. Sergio E. Oliva (ex auxiliar docente Gravimetr\'\i a, F.C.A.G). 
Desarrollo de algoritmos
para c\'alculo de anomal\'\i as gravitatorias 2D, durante 1995-1996. 

\medskip
Geof. Laura Mariana Longo (ex auxiliar docente M\'etodos Potenciales de 
Prospecci\'on, F.C.A.G). An\'alisis, implementaci\'on y aplicaci\'on de
deconvoluci\'on de Euler en anomal\'\i as magn\'eticas, 2002-2005. 
\bs

\noindent {\bf 21.4 Disc\'\i pulos de investigaci\'on con ubicaci\'on actual
\/}

\medskip

\noindent {\bf 21.5 Direcci\'on Personal de Apoyo a la Investigaci\'on.\/}
\bigskip

\noindent {\bf 22 - Antecedentes Profesionales Relevantes, Aportes
Significativos
a la Organizaci\'on Curricular:\/}

\bigskip

Actuaci\'on como Par Especialista para la evaluaci\'on de una  solicitud de 
promoci\'on en la carrera del Investigador de CONICET en el 
\'area Ciencias de la Tierra, en agosto de 2006.
%de la Dra. Silvina Geuna

Actuaci\'on como Par Especialista para la evaluaci\'on de tres solicitudes de 
promoci\'on a Investigador Independiente para CONICET,  
\'area Ciencias de la Tierra, en mayo de 2012.
% Geuna, Martinelli, Tomezzoli.

Actuaci\'on como Par Especialista para la evaluaci\'on de una solicitud de 
promoci\'on a Investigador Independiente para CONICET,  
\'area Ciencias de la Tierra, en abril de 2013.
% Zyserman.

Evaluaci\'on de un proyecto PICT-2013, \'Area de Ciencias F\'\i sicas, Matem\'aticas y Astron\'omicas, Fondo para la Investigaci\'on Cient\'\i fica y Tecnol\'ogica, Agencia Nacional de Promoci\'on Cient\'\i fica 
y Tecnol\'ogica, en octubre de 2013.

Evaluaci\'on de un proyecto PICT-2016, \'Area de Tecnolog\'\i a Energ\'etica 
Minera, Mec\'anica y de Materiales. FONCYT, Agencia Nacional de Promoci\'on Cient\'\i fica 
y Tecnol\'ogica, en Diciembre de 2016.

Actuaci\'on como Par Especialista para la evaluaci\'on de una solicitud de 
promoci\'on a Investigador Independiente para CONICET,  
\'area Ciencias de la Tierra, en abril de 2018.
%Vivien Gr�nhut

\bigskip

\noindent {\bf 23 - Direcci\'on de Institutos - Programas -
Laboratorios, etc.\/}



\end{document}
\enddocument


\grid
